\documentclass{article}

% External references
\usepackage{amsmath}
\usepackage{amssymb}
\usepackage{amsthm}
\usepackage{IEEEtrantools}
\usepackage{mathrsfs}
\usepackage{longtable}

% Mathematical elements
\newtheorem{theorem}{Theorem}
\newtheorem{corollary}{Corollary}
\newtheorem{lemma}{Lemma}
\newtheorem{observation}{Observation}
\newtheorem{proposition}{Proposition}
\theoremstyle{definition}\newtheorem{definition}{Definition}
\renewcommand{\IEEEproofindentspace}{0em}
\renewcommand{\IEEEQED}{\IEEEQEDopen}

\begin{document}
  \title{Gompertz Processes: A Theory of Ageing}
  \author{Aaron Geoffrey Sheldon}
  \date{\today}
  \maketitle

  \begin{abstract}
    Motivated by considering infinitesimal stochastic accelerations of time, we outline a
    theory of Gompertz processes, Poisson processes subordinated by integrated Geometric
    Brownian motion.
  \end{abstract}

  \section{Preliminaries}
  Basic science experiments in biology have ubiquitously observed that organisms respond to
  environmental stresses, including communicable diseases and exposures to toxins, with an
  acceleration $a$ in their failure time $a h\left(a t\right)$, where $h$ is the bare hazard
  rate of the failure event. Outside of the controlled setting of a laboratory the
  environmental stresses occur stochastically, resulting in a sequence of accelerations
  $a_n$ of the underlying metabolic time. The accelerations can either increase the rate of
  failure $a_n > 1$, an exacerbation of the stresses, or decrease the rate of failure
  $a_n < 1$, an alleviation from the stresses. However, even in the controlled setting of a
  laboratory it is experimentally challenging to directly measure the underlying accelerated
  metabolic time, instead we only have access to the failure time in bare units. Thus a
  theory of ageing must be one that studies the subordination of the failure time by an 
  elapsed metabolic time that is stochastically incremented.

  Over the course of an organism's lifetime it will encounter exacerbations and alleviations
  that accelerate $a_n,\dots,a_0=1$ metabolic time at times $t_n,\dots,t_0=0$. Furthermore,
  because the accelerations are all positive $a_i > 0$ for each acceleration we can find a
  finite real valued generator $b_n,\dots,b_0=0$ such that $a_i = e^{b_i}$. It follows that
  the elapsed metabolic time $\tilde{t}_{t_n}$ at time $t_n$ is the sum of the products of
  the accelerations up to time $t_{n-1}$ and the elapsed bare time steps $t_i - t_{i-1}$:
  \begin{IEEEeqnarray}{rCl}
    \tilde{t}_{t_n}
    & = & 
    \displaystyle \sum_{i=1}^{n}e^{\sum_{j=0}^{i-1} b_j} \left(t_i - t_{i-1}\right)
  \end{IEEEeqnarray}
  Provided no great explosions of acceleration occur in any small time scale, like say when
  an actual explosion occurs, the generators $b_i$ will become infinitesimal on the same
  order as $t_i - t_{i-1} \rightarrow 0$:
  \begin{IEEEeqnarray}{rCl}
    \mathcal{O}\left(b_i\right)
    & = & 
    \mathcal{O}\left(t_i - t_{i-1}\right)
  \end{IEEEeqnarray}
  The sum of products then becomes a stochastic process $\tilde{T}_t$ that is an integral of
  a geometric random infinitesimal generator process $e^{B_u}$:
  \begin{IEEEeqnarray}{rCl}
    \tilde{T}_t
    & = & 
    \int_0^t e^{B_u} du
  \end{IEEEeqnarray}
  This is essentially the continuous part of the Kolmorgorov's characterization of
  stochastic processes as being composed of either a finite number of discrete jumps or
  an infinite number of continuous changes in a span of time. If we assume as a first
  approximation that the exacerbations and alleviations, and their respective accelerations,
  are independent and stationary over time then by the L\'evy-Khintchine characterization 
  the only infinitesimal generator of elapsed metabolic time that is L\'evy and continuous,
  \emph{``jump free''}, is Brownian motion $B_u$. The stochastic process of elapsed metabolic 
  time is better known as integrated geometric Brownian motion.

  \begin{definition}[Gompertz Process]
    A Gompertz process $G_t$ is a subordinated Poisson process $N_t$, with rate $\lambda$,
    where the subordinating process is integrated geometric Brownian motion $Y_t$, with
    drift $\mu$ and diffusion $\sigma$:
    \begin{IEEEeqnarray}{rCl}
      G_t
      & = & 
      N_{Y_t}
    \end{IEEEeqnarray}
    given:
    \begin{IEEEeqnarray}{rCl}
      Y_t
      & = & 
      \int_0^t X_s ds\\
      & = &
      \int_0^t e^{\mu s + \sigma W_s} ds
    \end{IEEEeqnarray}
  \end{definition}
  Phenomenologically the finite real stochastic process $\mu t + \sigma W_s$ is the
  infinitesimal acceleration at time $t$ that generates a non-negative geometric stochastic
  process $X_t$ of accumulated accelerations up to time $t$ and whose integral $Y_t$ is a
  strictly increasing stochastic process of elapsed metabolic time up to time $t$.

  To start our exploration of the rich and subtleties of Gompertz processes we will briefly
  review of the properties of integrated geometric Brownian motion which are salient to
  developing our theory. This is by no means a comprehensive compendium. Much of the
  material I will cover has been deeply and thoroughly explored in the quantitative finance
  literature in the theory of pricing Asian options.
  
  Our first observation is that the increments of $Y_t$ can be factored by its carrier
  process $X_t$, for times $t > s$:
  \begin{IEEEeqnarray}{rCl}
    \operatorname{\mathbb{P}}\left[Y_t - Y_s\right]
    & = &
    \operatorname{\mathbb{P}}\left[X_s\right] \operatorname{\mathbb{P}}\left[Y_{t-s}\right]
  \end{IEEEeqnarray}
  where the process $X_s$ is independent of the process $Y_{t-s}$. For example, this allows
  us to immediately observe that for times $t > s$:
  \begin{IEEEeqnarray}{rCl}
    \operatorname{\mathbb{E}}\left[ \left(Y_t - Y_s\right)^n \right\rVert\left. X_s \right]
    & = &
    X_s^n \operatorname{\mathbb{E}}\left[ Y_{t-s}^n \right]
  \end{IEEEeqnarray}
  We will liberally exploit this technique of arbitraging the elapsed metabolic time $Y_t$ 
  against the accumulated acceleration $X_t$ to reduce expectations down to the well know
  standard terms for $X_t$ and $Y_t$:
  \begin{IEEEeqnarray}{rCl}
    \operatorname{\mathbb{E}}\left[ X_t^n \right]
    & = &
    e^{\left(n \mu + n^2 \sigma^2 / 2\right)t}\\
    \operatorname{\mathbb{E}}\left[ Y_t \right]
    & = &
    \frac{e^{\left(\mu + \sigma^2 / 2\right)t} - 1}{\mu + \sigma^2 / 2}
  \end{IEEEeqnarray}
  Note the in the last equation we have implicitly invoked Fubini's theorem to switch the
  order of integration, and will broadly continue to throughout this work.

  \begin{lemma}[Acceleration Lemma]
    The expectation of non-negative integer powers $n \ge 0 $ of the carrier process of
    geometric Brownian motion $X_t$ conditioned on the increment of integrated geometric
    Brownian motion $Y_t - Y_s$, where $t > s$, is given by:
    \begin{IEEEeqnarray}{rCl}
      \operatorname{\mathbb{E}}\left[ X_t^n \right\rVert\left. Y_t - Y_s \right]
      & = &
      \frac{\operatorname{\mathbb{E}}\left[X_t^n\right]}
      {\operatorname{\mathbb{E}}\left[\left(Y_t-Y_s\right)^n\right]}\left(Y_t-Y_s\right)^n
    \end{IEEEeqnarray}
  \end{lemma}
  \begin{proof}
    Carry out induction on $n$.
  \end{proof}

  A simple corollary follows from a nearly trivial derivation.

  \begin{corollary}[Counterpoint Corollary]
    For a triplet of times $t_1 > t_0 > t_{-1}$ we have the following conditional
    expectation:
    \begin{IEEEeqnarray}{rCl}
      \operatorname{\mathbb{E}}\left[ \left(Y_{t_1}-Y_{t_0}\right)^n \right\rVert\left. Y_{t_0} - Y_{t_{-1}} \right]
      & = &
      \displaystyle\frac{\operatorname{\mathbb{E}}\left[\left(Y_{t_1} - Y_{t_0}\right)^n\right]}
      {\operatorname{\mathbb{E}}\left[\left(Y_{t_0} - Y_{t_{-1}}\right)^n\right]}
      \left(Y_{t_0} - Y_{t_{-1}}\right)^n
    \end{IEEEeqnarray}
  \end{corollary}
  \begin{proof}
    Factor and apply the previous lemma.
  \end{proof}

  Even with the factorization observation we are still in need of a means of reducing the
  expectation of the powers $Y_t^n$. A small lemma suffices to provide the means of finding
  powers:

  \begin{lemma}[Recursion-Convolution Lemma]
    The expectation of non-negative integer powers $n \ge 0 $ of integrated geometric
    Brownian motion $Y_t$ is given by:
    \begin{IEEEeqnarray}{rCl}
      \operatorname{\mathbb{E}}\left[ Y_t^n \right]
      & = &
      n \int_0^t \operatorname{\mathbb{E}}\left[X_u^n\right] \operatorname{\mathbb{E}}\left[ Y_{t-u}^{n-1} \right] du
    \end{IEEEeqnarray}
  \end{lemma}
  \begin{proof}
    Carry out a routine factorization.
  \end{proof}

  With the preceding lemma in hand we have the sufficient tools required to estimate all the
  usual statistics involving powers of $Y_t$, including the expectation, variance, and
  covariances.

  \section{Martingale}
  By our counterpoint corollary the scaled increments of integrated geometric Brownian
  motion $Y_{s \uparrow t}$, with $t>s$:
  \begin{IEEEeqnarray}{rCl}
    Y_{s \uparrow t}
    & = &
    \displaystyle\frac
    {Y_t - Y_s}
    {\operatorname{\mathbb{E}}\left[ Y_t - Y_s \right]}
  \end{IEEEeqnarray}  
  form a two point martingale, so that for a triplet of times $t_1 > t_0 > t_{-1}$:
  \begin{IEEEeqnarray}{rCl}
    \operatorname{\mathbb{E}}\left[ Y_{t_0 \uparrow t_1} \right\rVert\left. Y_{t_{-1} \uparrow t_0} \right]
    & = &
    Y_{t_{-1} \uparrow t_0}
  \end{IEEEeqnarray}
  This allows us to leverage optional stopping time theorems to evaluate expectations of 
  stopped versions of Gompertz processes.

  \section{Markov}
  From the Martingale $Y_{s \uparrow t}$ we can infer that $Y_t$ is a 2 step Markov process.
  To verify this consider the sequence of bare times $0=t_0 < \dots < t_{n+1}=t$, and 
  accelerated times $0=y_0 < \dots < y_{n+1}=y$, working through the conditional probability
  we have:
  \begin{IEEEeqnarray}{rCl}
    \IEEEeqnarraymulticol{3}{l}{
      \operatorname{\mathbb{P}}\left[ Y_{t_{n+1}} = y_{n+1} \right\rVert\left. Y_{t_n} = y_n, \dots , Y_{t_0} = y_0 \right]
    }\nonumber\\
    & \qquad\qquad\qquad\qquad = &
    \int_0^\infty \int_0^\infty\operatorname{\mathbb{P}}\left[ X_{t_n - t_{n-1}} = \frac{u}{v}\frac{y_{n+1} - y_n}{y_n - y_{n-1}} \right]\nonumber\\
    & \qquad\qquad\qquad\qquad   &
    \qquad\qquad\qquad\qquad\cdot\operatorname{\mathbb{P}}\left[ Y_{t_n - t_{n-1}} = u \right]\nonumber\\
    & \qquad\qquad\qquad\qquad   &
    \qquad\qquad\qquad\qquad\qquad\cdot\operatorname{\mathbb{P}}\left[ Y_{t_{n+1} - t_n} = v \right] du dv\\
    & \qquad\qquad\qquad\qquad = &
    \operatorname{\mathbb{P}}\left[ Y_{t_{n+1}} = y_{n+1} \right\rVert\left. Y_{t_n} = y_n, Y_{t_{n-1}} = y_{n-1} \right]
  \end{IEEEeqnarray}
  Thus any triple of stopping times $T_{n+1},T_n,T_{n-1}$ of consecutive passages of $G_t$
  will have a $2$ step Markov dependence. This amounts to a concrete prediction that ageing
  alone introduces a statistical dependence between the intervals of consecutive admissions
  for healthcare services.

  \section{L\'evy}
  Integrated geometric Brownian motion $Y_t$ and its carrier process of geometric Brownian
  motion $X_t$ can be embedded into the Lie algebra of $2 \times 2$ upper triangular
  matrices $\mathfrak{h}_2$ by means of the factorization observed earlier, so that the
  increments are independent under matrix multiplication, for times $t > s$:
  \begin{IEEEeqnarray}{rCl}
    \begin{bmatrix}
      1 & Y_t\\
      0 & X_t
    \end{bmatrix}
    & \sim &
    \begin{bmatrix}
      1 & \tilde{Y}_{t-s}\\
      0 & \tilde{X}_{t-s}
    \end{bmatrix}
    \begin{bmatrix}
      1 & Y_s\\
      0 & X_s
    \end{bmatrix}
  \end{IEEEeqnarray}
  where the increment processes of $\tilde{Y}_{t-s}$ and $\tilde{X}_{t-s}$ are independent
  of the processes $Y_s$ and $X_s$.

  \section{Fokker-Planck}
  The upper triangular L\'evy process of integrated geometric Brownian motion satisfies the
  stochastic differential equation:
  \begin{IEEEeqnarray}{rCl}
    d \begin{bmatrix}
      1 & Y_t\\
      0 & X_t
    \end{bmatrix}
    & = &
    \begin{bmatrix}
      0 & X_t\\
      0 & \left(\mu + \frac{\sigma^2}{2}\right) X_t
    \end{bmatrix} dt
    +
    \begin{bmatrix}
      0 & 0\\
      0 & \sigma X_t
    \end{bmatrix}dW_t
  \end{IEEEeqnarray}
  It follows from Fokker-Planck that the probability density of the joint process 
  $p=\operatorname{\mathbb{P}}\left[Y_t=y,X_t=x\right]$ satisfies
  the partial differential equation:
  \begin{IEEEeqnarray}{rCl}
    \frac{\partial p}{\partial t}
    & = &
    - \frac{\partial}{\partial x} \left(\mu + \frac{\sigma^2}{2}\right)xp
    - \frac{\partial}{\partial y} x p
    + \frac{\partial^2}{\partial x^2} \frac{\sigma^2}{2}x^2 p
  \end{IEEEeqnarray}
  We can restate this as an Eigen evolution equation:
  \begin{IEEEeqnarray}{rCl}
    \frac{\partial p}{\partial t}
    + \left(\mu + \frac{3}{2}\sigma^2\right) x \frac{\partial p}{\partial x}
    + x \frac{\partial p}{\partial y}
    - \frac{\sigma^2}{2}x^2 \frac{\partial^2 p}{\partial x^2}
    & = &
    \left(-\mu + \frac{\sigma^2}{2}\right) p
  \end{IEEEeqnarray}
  Marginalizing over the probability of $X_t$ and applying the acceleration lemma
  yields the first order partial differential equation for the distribution of $Y_t$:
  \begin{IEEEeqnarray}{rCl}
    \frac{\operatorname{\mathbb{E}}\left[ Y_t \right]}
    {\operatorname{\mathbb{E}}\left[ X_t \right]}
    \frac{\partial p}{\partial t}
    - y\frac{\partial p}{\partial y}
    & = &
    p
  \end{IEEEeqnarray}
  Which by trial solution of separation of variables has the general solution:
  \begin{IEEEeqnarray}{rCl}
    p
    & = &
    \operatorname{\mathbb{E}}\left[ Y_t \right]
    f_{\mu, \sigma}\left(y \operatorname{\mathbb{E}}\left[ Y_t \right]\right)
  \end{IEEEeqnarray}
  for any analytic $f_{\mu, \sigma}$ dependent on the drift and diffusion of the carrier
  process $X_t$. Note that we are offloading the dimensional analysis into the analytic
  function.

  \section{Hazard Rate}
  Consider the first passage stopping time $T_1$ of the Gompertz process $G_t$, its 
  cumulative distribution is the characteristic function of $Y_t$:
  \begin{IEEEeqnarray}{rCl}
    \operatorname{\mathbb{P}}\left[ T_1 \ge t\right]
    & = &
    \operatorname{\mathbb{E}}\left[ e^{-\lambda Y_t} \right]
  \end{IEEEeqnarray}
  Thus the hazard rate $h$ of $T_1$:
  \begin{IEEEeqnarray}{rCl}
    h
    & = &
    \operatorname{\mathbb{P}}\left[ T_1 = t\right\rVert\left. T_1 \ge t\right]
  \end{IEEEeqnarray}
  satisfies the partial differential Eigen equation:
  \begin{IEEEeqnarray}{rCl}
    \frac{\operatorname{\mathbb{E}}\left[Y_t\right]}{\operatorname{\mathbb{E}}\left[X_t\right]}
    \frac{\partial h}{\partial t} - \lambda\frac{\partial h}{\partial \lambda}
    & = &
    \left(\frac{\operatorname{\mathbb{E}}\left[Y_t\right]}{\operatorname{\mathbb{E}}\left[X_t\right]}\right)^2
    \left(\frac{\partial}{\partial t}\frac{\operatorname{\mathbb{E}}\left[X_t\right]}{\operatorname{\mathbb{E}}\left[Y_t\right]}\right) h
  \end{IEEEeqnarray}
  Which by trial solution of separation of variables has the general solution:
  \begin{IEEEeqnarray}{rCl}
    h
    & = &
    \lambda^2 \operatorname{\mathbb{E}}\left[X_t\right]
    \operatorname{\mathbb{E}}\left[Y_t\right]
    g_{\mu, \sigma}\left( \lambda \operatorname{\mathbb{E}}\left[Y_t\right] \right)
  \end{IEEEeqnarray}
  for any analytic $g_{\mu, \sigma}$ dependent on the drift and diffusion of
  the carrier process $X_t$. Taking the limit to deterministic subordination yields the
  constraints on $g_{\mu, \sigma}$:
  \begin{longtable}{lll}
    \caption{Sequential Boundary Conditions}\\
    \multicolumn{1}{l}{Boundary} & \multicolumn{1}{l}{Condition} & \multicolumn{1}{l}{Removes}\\
    \hline
    \endfirsthead
    \caption*{Continued from previous page.}\\
    \multicolumn{1}{l}{Boundary} & \multicolumn{1}{l}{Condition} & \multicolumn{1}{l}{Removes}\\
    \hline
    \endhead
    \caption*{Continued on next page.}
    \endfoot
    \caption*{Sequential boundary conditions on the hazard rate $h$ derived from the limits to deterministic subordination.}
    \endlastfoot\\
    $\sigma=0$    & $h=\lambda e^{\mu t}$ & diffusion\\\\
    $\mu=0$       & $h=\lambda$           & then drift\\\\
    $\lambda = 0$ & $h=0$                 & finally jumps
  \end{longtable}
  From the boundary conditions we can immediately deduce that in the deterministic limit of
  $\sigma \rightarrow 0$ we have:
  \begin{IEEEeqnarray}{rCl}
    g_{\mu, \sigma}\left(x\right)
    & \xrightarrow{\sigma = 0} &
    \frac{1}{x}
  \end{IEEEeqnarray}
  However this alone cannot be the solution as it results in the characteristic function in
  $\lambda$ of a purely deterministic $Y_t$. Equating the general solution for the hazard
  rate to the Laplace of the Fokker-Planck solution yields the implicit equation in
  $f_{\mu,\sigma}$ and $g_{\mu,\sigma}$:
  \begin{IEEEeqnarray}{rCl}
    \displaystyle-\frac{\partial}{\partial t} \ln
    \int_0^\infty
    e^\frac{\lambda u}{\operatorname{\mathbb{E}}\left[Y_t\right]}
    f_{\mu,\sigma}\left(u\right)
    du
    & = &
    \lambda^2
    \operatorname{\mathbb{E}}\left[X_t\right]
    \operatorname{\mathbb{E}}\left[Y_t\right]
    g_{\mu, \sigma}\left( \lambda \operatorname{\mathbb{E}}\left[Y_t\right] \right)
  \end{IEEEeqnarray} 
  Dimensional analysis provides an inference for a solution, which remains an open problem:
  \begin{proposition}[Gompertz Anomaly]
    The analytic function $g_{\mu, \sigma}$ is simply the exponential divided by its
    argument, so that the hazard $h$ is given by:
    \begin{IEEEeqnarray}{rCl}
      h
      & = &
      \lambda \operatorname{\mathbb{E}}\left[X_t\right]
      e^{\frac{\sigma^2 / 2}{\mu + \sigma^2 / 2}  \lambda\operatorname{\mathbb{E}}\left[Y_t\right]}
    \end{IEEEeqnarray}
  \end{proposition}
  \begin{proof}
    The parity multiplied derivatives of the exponential of the integral of the hazard rate
    satisfies the recursion-convolution lemma, generating the moments of $Y_t$, and hence is
    the characteristic function of $Y_t$.
  \end{proof}

  \section{Discussion}
  In most circumstances $\lambda$ is the new born infant mortality due to ageing alone, and
  is less than $1$ in $32000$ person-years. As such the hazard is very close to the original
  hazard observed by Gompertz. Intuitively, when $\mu \ggg \sigma^2 / 2$ the process becomes
  approximately deterministic due to the large impact of the drift.

  Conversely, the central conjecture in the preliminary material is that ageing is driven by
  stochastic accelerations, hence requiring that the drift vanish $\mu = 0$. In this case
  the anomalous Gompertz hazard rate simplifies to:
  \begin{IEEEeqnarray}{rCl}
    h
    & = &
    \lambda e^{t\sigma^2 / 2}
    e^{\lambda \frac{e^{t\sigma^2 / 2} - 1}{\sigma^2 / 2}}
  \end{IEEEeqnarray}
  We have observed the anomalous Gompertz hazard rate in mortality rates in Alberta in the
  Hazard Rate Zoo experiment. Specifically when we remove the dominate exponential process
  of a doubling of mortality every $7$ years from an infant mortality of $1$ in $32000$
  person-years:
  \begin{IEEEeqnarray}{rCl}
    h
    & = &
    \displaystyle\frac{2^{t/7}}{32000}
    e^\frac{7 \left(2^{t/7} - 1\right)}{32000 \ln 2}
  \end{IEEEeqnarray}
  there remains a residual anomalous growth in mortality with ageing, reflecting the higher
  order affect of the stochastic accelerations.
\end{document}