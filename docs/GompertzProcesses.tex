\documentclass{article}

% External references
\usepackage{amsmath}
\usepackage{amssymb}
\usepackage{amsthm}
\usepackage{IEEEtrantools}
\usepackage{mathrsfs}
\usepackage{longtable}

% Mathematical elements
\newtheorem{theorem}{Theorem}
\newtheorem{corollary}{Corollary}
\newtheorem{lemma}{Lemma}
\newtheorem{observation}{Observation}
\newtheorem{proposition}{Proposition}
\theoremstyle{definition}\newtheorem{definition}{Definition}
\renewcommand{\IEEEproofindentspace}{0em}
\renewcommand{\IEEEQED}{\IEEEQEDopen}

\begin{document}
  \title{Gompertz Processes: A Theory of Ageing}
  \author{Aaron Geoffrey Sheldon}
  \date{\today}
  \maketitle

  \begin{abstract}
    Ageing is a accelerating failure process that is universal to all biological systems.
    Motivated by considering infinitesimal stochastic accelerations of time, we hypothesize
    a general explanation for the emergent determinism of ageing processes in the theory of
    Gompertz processes: Poisson processes subordinated by integrated Geometric Brownian
    motion.
  \end{abstract}

  \section{Preliminaries}
  Basic science experiments in biology have ubiquitously observed that organisms respond to
  environmental stresses, including communicable diseases and exposures to toxins, with an
  acceleration $a$ in their failure time $a h\left(a t\right)$, where $h\left(t\right)$ is
  the unperturbed hazard rate of the failure event. Outside of the controlled setting of a
  laboratory the environmental stresses occur stochastically, resulting in a stochastic
  sequence of accelerations $a_n$ of the organism's metabolic time. The stochastic
  accelerations can either increase the rate of failure $a_n > 1$, an exacerbation of the
  stresses, or decrease the rate of failure $a_n < 1$, an alleviation from the stresses.
  However, even in the controlled setting of a laboratory it is experimentally challenging
  to directly measure the stochastically accelerated metabolic time of the organism, instead
  we only have access to the failure events measured at bare time units. Thus a theory of
  ageing must study the subordination of the failure time by a stochastically accelerated
  metabolic time.

  Over the course of an organism's lifetime it will encounter exacerbations and alleviations
  that stochastically accelerate $a_n,\dots,a_0=1$ metabolic time at times
  $t_n,\dots,t_0=0$. Furthermore, because the stochastic accelerations are all positive
  $a_i > 0$ for each stochastic acceleration we can find a finite real valued generator
  $w_n,\dots,w_0=0$ such that $a_i = e^{w_i}$. It follows that the stochastically
  accelerated metabolic time $y_{t_n}$ at time $t_n$ is the sum of the products of the
  stochastic accelerations up to time $t_{n-1}$ and the elapsed bare time steps
  $t_i - t_{i-1}$:
  \begin{IEEEeqnarray}{rCl}
    y_{t_n}
    & = & 
    \displaystyle \sum_{i=1}^{n}e^{\sum_{j=0}^{i-1} w_j} \left(t_i - t_{i-1}\right)
  \end{IEEEeqnarray}
  Provided no great explosions of acceleration occur in any small time scale, like say when
  an actual explosion occurs, the generators $w_i$ will become infinitesimal on the same
  order as the bare time steps become infinitesimal $t_i - t_{i-1} \rightarrow 0$:
  \begin{IEEEeqnarray}{rCl}
    \mathcal{O}\left(w_i\right)
    & = & 
    \mathcal{O}\left(t_i - t_{i-1}\right)
  \end{IEEEeqnarray}
  The sum of products then becomes a stochastic process $Y_t$ that is an integral of
  a random geometric infinitesimal generator process $e^{W_u}$:
  \begin{IEEEeqnarray}{rCl}
    Y_t
    & = & 
    \int_0^t e^{W_u} du
  \end{IEEEeqnarray}

  This asymptotic argument is essentially the continuous part of the Kolmorgorov's
  characterization of stochastic processes as being composed, in bounded time, of either a
  finite number of discrete jumps or an infinite number of continuous changes. If we assume
  as a first approximation that the exacerbations and alleviations, and their respective
  stochastic accelerations, are independent and stationary over time then by the
  L\'evy-Khintchine characterization the only infinitesimal generator of stochastically
  accelerated metabolic time that is L\'evy and continuous \emph{``jump free''} is Brownian
  motion $W_u$. The stochastically accelerated metabolic time is better known as integrated
  geometric Brownian motion.

  \section{Gompertz}
  Motivated by the preceding heuristic derivation we formally define the Gompertz process on
  time $t \ge 0$.
  \begin{definition}[Gompertz Process]
    A Gompertz process $G_t$ is a subordinated Poisson process $N_t$, with rate $\lambda$,
    where the subordinating process is integrated geometric Brownian motion $Y_t$, with
    drift $\mu$ and diffusion $\sigma$:
    \begin{IEEEeqnarray}{rCl}
      G_t
      & = & 
      N_{Y_t}
    \end{IEEEeqnarray}
    given:
    \begin{IEEEeqnarray}{rCl}
      Y_t
      & = & 
      \int_0^t X_s ds\\
      & = &
      \int_0^t e^{\mu s + \sigma W_s} ds
    \end{IEEEeqnarray}
  \end{definition}
  Phenomenologically the finite real stochastic process $\mu t + \sigma W_s$ is the
  infinitesimal acceleration at time $t$ that generates a non-negative geometric stochastic
  process $X_t$ of accumulated accelerations up to time $t$ and whose integral $Y_t$ is a
  strictly increasing stochastically accelerated metabolic time up to time $t$.

  From the definition of the Gompertz process $G_t$ conditioning on the history of a sample
  path of integrated geometric Brownian motion $Y_t$ yields a conditional Poisson process:
  \begin{IEEEeqnarray}{rCl}
    \operatorname{\mathbb{P}}\left[G_t = n \right\rVert\left. Y_t \right]
    & = & 
    \frac{\left(\lambda Y_t\right)^n}{n!} e^{-\lambda Y_t}\\\nonumber\\
    \operatorname{\mathbb{E}}\left[G_t \right\rVert\left. Y_t \right]
    & = &
    \lambda Y_t
  \end{IEEEeqnarray}
  Dual to conditioning integrated geometric Brownian motion, conditioning on the Gompertz
  process estimates the elapsed stochastically accelerated metabolic time:
  \begin{IEEEeqnarray}{rCl}
    \operatorname{\mathbb{P}}\left[Y_t \le y \right\rVert\left. G_t \right]
    & = & 
    \displaystyle\int_0^{\lambda y} \frac{u^{G_t - 1} e^{-u}}{\left(G_t - 1\right)!} du\\\nonumber\\
    \operatorname{\mathbb{E}}\left[Y_t \right\rVert\left. G_t \right]
    & = &
    \frac{G_t}{\lambda}
  \end{IEEEeqnarray}
  Reflecting that integrated geometric Brownian motion $Y_t$ is truly measuring the elapsed
  stochastically accelerated metabolic time.

  To start our exploration of the rich subtleties of the Gompertz process we will briefly
  review the properties of integrated geometric Brownian motion which are salient to
  developing our theory. This is by no means a comprehensive compendium. Much of the
  material I will cover has been deeply and thoroughly explored in the quantitative finance
  literature in the theory of pricing Asian options, and the stochastic processes syllabus
  on subordinated Poisson processes, usually in the context of deriving the characteristic
  function of the subordinating process.
  
  Our first observation is that the increments of $Y_t$ can be factored by its carrier
  process $X_t$, for times $t > s$:
  \begin{IEEEeqnarray}{rCl}
    \operatorname{\mathbb{P}}\left[Y_t - Y_s\right]
    & = &
    \operatorname{\mathbb{P}}\left[X_s\right] \operatorname{\mathbb{P}}\left[Y_{t-s}\right]
  \end{IEEEeqnarray}
  where the process $X_s$ is independent of the process $Y_{t-s}$. Phenomenologically the
  increments of $Y_t - Y_s$ are equivalent to a process that starts with acceleration $X_s$.
  Analogous to the Fundamental Theorem of Arithmetic this factorization allows us to reach
  many useful inferences; for example we can immediately observe that for times $t > s$:
  \begin{IEEEeqnarray}{rCl}
    \operatorname{\mathbb{E}}\left[ \left(Y_t - Y_s\right)^n \right\rVert\left. X_s \right]
    & = &
    X_s^n \operatorname{\mathbb{E}}\left[ Y_{t-s}^n \right]
  \end{IEEEeqnarray}
  We will liberally exploit this technique of arbitraging the stochastically accelerated
  metabolic time $Y_t$ against the accumulated stochastic accelerations $X_t$ to reduce
  expectations down to the well known standard terms for $X_t$ and $Y_t$:
  \begin{IEEEeqnarray}{rCl}
    \operatorname{\mathbb{E}}\left[ X_t^n \right]
    & = &
    e^{\left(n \mu + n^2 \sigma^2 / 2\right)t}\\\nonumber\\
    \operatorname{\mathbb{E}}\left[ Y_t \right]
    & = &
    \frac{e^{\left(\mu + \sigma^2 / 2\right)t} - 1}{\mu + \sigma^2 / 2}
  \end{IEEEeqnarray}
  Note that in the last equation we have implicitly invoked the Fubini-Tonelli Theorem to
  switch the order of integration, and will broadly continue to use this theorem throughout
  this work.

  \begin{lemma}[Acceleration Lemma]
    The expectation of non-negative integer powers $n \ge 0 $ of the carrier process of
    geometric Brownian motion $X_t$ conditioned on integrated geometric Brownian motion 
    $Y_t$ are given by:
    \begin{IEEEeqnarray}{rCl}
      \operatorname{\mathbb{E}}\left[ X_t^n \right\rVert\left. Y_t \right]
      & = &
      \frac{\operatorname{\mathbb{E}}\left[X_t^n\right]}
      {\operatorname{\mathbb{E}}\left[Y_t^n\right]}Y_t^n
    \end{IEEEeqnarray}
  \end{lemma}
  \begin{proof}
    We proceed by induction on $n$.
    \begin{enumerate}
      \item For $n=0$ clearly $\operatorname{\mathbb{E}}\left[ X_t^0 \right\rVert\left. Y_t\right] = 1$
      \item Now assume that up to $n$ the relation holds.
      \item For $u < t$ consider the integrable function
      \begin{IEEEeqnarray}{rCl}
        f\left(u, t, y\right)
        & = &
        \operatorname{\mathbb{E}}\left[ X_u X_t^n \right\rVert\left. Y_t = y \right]
      \end{IEEEeqnarray}
      \item It follows from the Fubini-Tonelli Theorem and the induction assumption that for
        $n$ the integral is
        \begin{IEEEeqnarray}{rCl}
          \int_0^t f\left(u, t, y\right) du
          & = &
          y^{n+1} \frac{\operatorname{\mathbb{E}}\left[X_t^n\right]}
          {\operatorname{\mathbb{E}}\left[Y_t^n\right]}
        \end{IEEEeqnarray}
      \item Thus by the Fundamental Theorem of Calculus the integrand has the decomposition
        $f\left(u, t, y\right) = y^{n+1} f\left(u, t\right)$
      \item By the Law of Conditional Expectations we have
        \begin{IEEEeqnarray}{rCl}
          \operatorname{\mathbb{E}}\left[ X_u X_t^n\right]
          & = &
          \operatorname{\mathbb{E}}\left[\operatorname{\mathbb{E}}\left[ X_u X_t^n \right\rVert\left. Y_t \right]\right]\\
          & = &
          f\left(u, t\right) \operatorname{\mathbb{E}}\left[ Y_t^{n+1}\right]
        \end{IEEEeqnarray}
      \item Which immediately yields the decomposed integrand as
        \begin{IEEEeqnarray}{rCl}
          f\left(u,t\right)
          & = &\frac{\operatorname{\mathbb{E}}\left[X_u X_t^n\right]}
          {\operatorname{\mathbb{E}}\left[Y_t^{n+1}\right]}
        \end{IEEEeqnarray}
      \item Finally taking $u=t$ yields the desired result
        \begin{IEEEeqnarray}{rCl}
          \operatorname{\mathbb{E}}\left[ X_t^{n+1} \right\rVert\left. Y_t \right]
          & = &
          f\left(s,t,t\right)Y_t^{n+1}\\
          & = &
          \frac{\operatorname{\mathbb{E}}\left[X_t^{n+1}\right]}
          {\operatorname{\mathbb{E}}\left[Y_t^{n+1}\right]}Y_t^{n+1}
        \end{IEEEeqnarray}
    \end{enumerate}
  \end{proof}

  A simple corollary follows from a nearly trivial derivation.

  \begin{corollary}[Counterpoint Corollary]
    For a times $t > s$ we have the following conditional
    expectation:
    \begin{IEEEeqnarray}{rCl}
      \operatorname{\mathbb{E}}\left[ Y_t^n \right\rVert\left. Y_s \right]
      & = &
      \displaystyle\frac{\operatorname{\mathbb{E}}\left[Y_t^n\right]}
      {\operatorname{\mathbb{E}}\left[Y_s^n\right]}
      Y_s^n
    \end{IEEEeqnarray}
  \end{corollary}
  \begin{proof}
    Factoring and applying the previous lemma:
    \begin{IEEEeqnarray}{rCl}
      \operatorname{\mathbb{E}}\left[Y_t^n \right\rVert\left. Y_s \right]
      & = &
      \operatorname{\mathbb{E}}\left[ X_s^n \right\rVert\left. Y_s \right]\operatorname{\mathbb{E}}\left[ Y_{t-s}^n \right]\\
      & = &
      \frac{\operatorname{\mathbb{E}}\left[ Y_{t-s}^n \right]\operatorname{\mathbb{E}}\left[X_s^n\right]}
      {\operatorname{\mathbb{E}}\left[Y_s^n\right]}Y_s^n\\
      & = &
      \frac{\operatorname{\mathbb{E}}\left[Y_t^n\right]}
      {\operatorname{\mathbb{E}}\left[Y_s^n\right]}
      Y_s^n
    \end{IEEEeqnarray}
  \end{proof}

  Even with the factorization observation we are still in need of a means of reducing the
  expectation of the powers $Y_t^n$. A small lemma suffices to provide the means of finding
  powers:

  \begin{lemma}[Recursion-Convolution Lemma]
    The expectation of non-negative integer powers $n \ge 0 $ of integrated geometric
    Brownian motion $Y_t$ is given by:
    \begin{IEEEeqnarray}{rCl}
      \operatorname{\mathbb{E}}\left[ Y_t^n \right]
      & = &
      n \int_0^t \operatorname{\mathbb{E}}\left[X_u^n\right] \operatorname{\mathbb{E}}\left[ Y_{t-u}^{n-1} \right] du
    \end{IEEEeqnarray}
  \end{lemma}
  \begin{proof}
    Expanding the expectant as a multi-variable integral we have, applying th Fubini-Tonelli
    Theorem we have, and then factoring we have:
    \begin{IEEEeqnarray}{rCl}
      \operatorname{\mathbb{E}}\left[ Y_t^n \right]
      & = &
      \int_0^t \dots \int_0^t \operatorname{\mathbb{E}}\left[ X_t^n \right] du_1 \dots du_n\\
      & = &
      \binom{n}{1} \int_0^t \operatorname{\mathbb{E}}\left[ X_u \left(Y_t - Y_u\right)^{n-1} \right] du\\
      & = &
      n \int_0^t \operatorname{\mathbb{E}}\left[ X_u^n\right] \operatorname{\mathbb{E}}\left[Y_{t-u}^{n-1} \right] du
    \end{IEEEeqnarray}
  \end{proof}

  With the preceding lemma in hand we have the sufficient tools required to estimate all the
  usual statistics involving powers of $Y_t$, including the expectation, variance, and
  covariances.

  \section{Martingale}
  By our counterpoint corollary the amortized integrated geometric Brownian motion:
  \begin{IEEEeqnarray}{rCl}
    \hat{Y}_t
    & = &
    \displaystyle\frac
    {Y_t}
    {\operatorname{\mathbb{E}}\left[ Y_t \right]}
  \end{IEEEeqnarray}  
  forms a martingale, so that for times $t > s$:
  \begin{IEEEeqnarray}{rCl}
    \operatorname{\mathbb{E}}\left[ \hat{Y}_t \right\rVert\left. \hat{Y}_s \right]
    & = &
    \hat{Y}_s
  \end{IEEEeqnarray}
  We can deduce that the amortized Gompertz process:
  \begin{IEEEeqnarray}{rCl}
    \hat{G}_t
    & = &
    \displaystyle\frac
    {G_t}
    {\operatorname{\mathbb{E}}\left[ Y_t \right]}
  \end{IEEEeqnarray}
  is a martingale, so that for times $t  > s$:
  \begin{IEEEeqnarray}{rCl}
    \operatorname{\mathbb{E}}\left[ \hat{G}_t \right\rVert\left. \hat{G}_s \right]
    & = &
    \hat{G}_s
  \end{IEEEeqnarray}
  This allows us to leverage optional stopping time theorems to evaluate expectations of 
  stopped versions of Gompertz processes.

  \section{Markov}
  Even though amortized integrated geometric Brownian motion $\hat{Y}_t$ is a martingale,
  integrated geometric Brownian motion $Y_t$ is a 2 step Markov process, and thus the
  increments $Y_t - Y_s$, with $t > s$ are Markov. To verify this consider the sequence of
  bare times $0=t_0 < \dots < t_{n+1}=t$, and accelerated times $0=y_0 < \dots < y_{n+1}=y$,
  working through the conditional probability we have:
  \begin{IEEEeqnarray}{rCl}
    \IEEEeqnarraymulticol{3}{l}{
      \operatorname{\mathbb{P}}\left[ Y_{t_{n+1}} - Y_{t_n} = y_{n+1} - y_n \right\rVert\left. Y_{t_n} - Y_{t_{n-1}} = y_n - y_{n-1}, \dots , Y_{t_1} - Y_{t_0} = y_1 - y_0 \right]
    }\nonumber\\
    & \qquad\qquad\qquad = &
    \int_0^\infty \int_0^\infty\operatorname{\mathbb{P}}\left[ X_{t_n - t_{n-1}} = \frac{u}{v}\frac{y_{n+1} - y_n}{y_n - y_{n-1}} \right]\nonumber\\
    & \qquad\qquad\qquad   &
    \qquad\qquad\qquad\qquad\cdot\operatorname{\mathbb{P}}\left[ Y_{t_n - t_{n-1}} = u \right]\nonumber\\
    & \qquad\qquad\qquad   &
    \qquad\qquad\qquad\qquad\qquad\cdot\operatorname{\mathbb{P}}\left[ Y_{t_{n+1} - t_n} = v \right] du dv\\
    & \qquad\qquad\qquad = &
    \operatorname{\mathbb{P}}\left[ Y_{t_{n+1}} - Y_{t_n} = y_{n+1} - y_n \right\rVert\left.  Y_{t_n} - Y_{t_{n-1}} = y_n - y_{n-1} \right]
  \end{IEEEeqnarray}
  We can conveniently derive from the martingale the conditional expectation of the
  increment for a triple of times $t_1 > t_0 > t_{-1}$:
  \begin{IEEEeqnarray}{rCl}
    \operatorname{\mathbb{E}}\left[ Y_{t_1} - Y_{t_0} \right\rVert\left. Y_{t_0} - Y_{t_{-1}} \right]
    & = &
    \frac{\operatorname{\mathbb{E}}\left[ Y_{t_1} - Y_{t_0}\right]}
    {\operatorname{\mathbb{E}}\left[Y_{t_0} - Y_{t_{-1}} \right]}
    \left(Y_{t_0} - Y_{t_{-1}}\right)
  \end{IEEEeqnarray}
  In practice we cannot directly measure the stochastically accelerated metabolic time $Y_t$
  instead we have access to the stopping times $T_n$ of the passages $G_{T_n} = n$ of the
  Gompertz process.
  
  To study the Markov properties of the stopping times of the passages of the Gompertz
  process we revisit the recovery of the Poisson process from the Gompertz process by
  conditioning on the history of the sample path of integrated geometric Brownian motion.
  By differentiating the subordinated cumulative distribution of a single stopping time we
  have the conditional probability distribution:
  \begin{IEEEeqnarray}{rCl}
    \operatorname{\mathbb{P}}\left[ T_n = t \right\rVert\left. Y_{T_n} \right]
    & = &
    \frac{\operatorname{\mathbb{E}}\left[ X_t \right]}{\operatorname{\mathbb{E}}\left[ Y_t \right]}
    \frac{\left(\lambda Y_t\right)^n}{\left(n-1\right)!}
    e^{-\lambda Y_t}
  \end{IEEEeqnarray}
  Finding a closed form for the distribution of the stopping times of the passages of the
  Gompertz process will require a deeper understanding of the characteristic function of
  integrated geometric Brownian motion, which we will develop later. In the meantime there
  are many fruits to be plucked from a study of stopping times of the passages of the
  Gompertz process.

  The central statistic of study in the longitudinal analysis of biological systems is the
  latency $T_{1+G_t} - T_{G_t} =T_{1+G_t} -t$ between consecutive stopping times of the
  passages of the Gompertz process. Due to immortal time bias
  \emph{``we cannot see indefinitely into the past''} we have subordinated the stopping
  times $T_{n+G_t}$ of the passages of the Gompertz process by the increments of the
  Gompertz process $n+G_t$ from a sentinel event $G_t$. In practice observational studies,
  particularly in clinical research and epidemiology, are only able to observe consecutive
  passages of the Gompertz process from a fixed sentinel event without the knowledge of how
  many events have occurred before the sentinel event.
  
  From the preceding probability density we can immediately deduce the tail probability and
  hence the expectation of the latency $s>0$ conditioned on a sentinel event at time $t>0$:
  \begin{IEEEeqnarray}{rCl}
    \operatorname{\mathbb{P}}\left[ T_{1+G_t} - T_{G_t} \ge s \right\rVert\left. T_{G_t}=t \right]
    & = &
    \operatorname{\mathbb{E}}\left[ e^{-\lambda X_t Y_s } \right]\\\nonumber\\
    \operatorname{\mathbb{E}}\left[T_{1+G_t} - T_{G_t}  \right\rVert\left. T_{G_t}=t \right]
    & = &
    \int_0^\infty \operatorname{\mathbb{E}}\left[ e^{-\lambda X_t Y_s} \right] ds
  \end{IEEEeqnarray}
  where $Y_s$ is the increment process independent of the acceleration $X_t$ at the start of
  the increment. 
  
  Remarkably the stopping times of the passages for the Gompertz process are Markov, even
  though the stochastically accelerated metabolic time of integrated geometric Brownian
  motion is only Markov in its increments. Specifically knowledge of the age $t_0>0$ at a
  sentinel event $G_{t_0}$ in the history of the Gompertz process is sufficient to determine
  the distribution of the latency $t > 0$ to next event $T_{1+n+G_{t_0}}$. To see this
  consider the subordinated stopping times $T_{n+G_{t_0}}=t_n, \dots , T_{G_{t_0}} = t_0$,
  the cumulative probability of the latency conditioned of the previous events is:
  \begin{IEEEeqnarray}{rCl}
    \IEEEeqnarraymulticol{3}{l}{
      \operatorname{\mathbb{P}}\left[ T_{1 + n + G_t} - T_{n + G_t} \ge t \right\rVert\left. T_{n+G_{t_0}}=t_n, \dots , T_{G_{t_0}} = t_0 \right]
    }\nonumber\\
    \qquad
    & = &
    \operatorname{\mathbb{E}}\left[ e^{-\lambda \left(Y_{t + t_n} - Y_{t_n}\right)} \right\rVert\left. T_{n+G_{t_0}}=t_n, \dots , T_{G_{t_0}} = t_0 \right]\\
    \qquad
    & = &
    \operatorname{\mathbb{E}}\left[ e^{-\lambda X_{t_n} Y_t}\right]\\
    \qquad
    & = &
    \operatorname{\mathbb{P}}\left[ T_{1 + G_{t_n}} - T_{G_{t_n}} \ge t \right\rVert\left. T_{G_{t_n}}=t_n \right]
  \end{IEEEeqnarray}
  where $Y_t$ is the increment process independent of the acceleration $X_{t_n}$ at the
  start of the increment.

  Given that the stopping times of consecutive passages of the Gompertz process are Markov
  we consider the covariance of the latency between consecutive events
  $T_{2+G_t},T_{1+G_t}, T_{G_t}$ conditioned on the sentinel event $T_{G_t}=t$:
  \begin{IEEEeqnarray}{rCl}
    \IEEEeqnarraymulticol{3}{l}{
      \operatorname{\mathbb{C}ov}\left[ T_{2+G_t} - T_{1+G_t}, T_{1+G_t} - T_{G_t} \right\rVert\left. T_{G_t}=t \right]
    }\nonumber\\
    \qquad
    & = &
    \displaystyle\int_0^\infty\displaystyle\int_0^\infty
    \operatorname{\mathbb{E}}\left[ e^{-\lambda X_{t+v} Y_u}\right]
    \operatorname{\mathbb{E}}\left[\lambda v X_v e^{-\lambda X_t Y_v}\right]
    dv du\nonumber\\
    &   &
    \qquad- \displaystyle\int_0^\infty \operatorname{\mathbb{E}}\left[\lambda X_t Y_u e^{-\lambda X_t Y_u}\right] du
    \displaystyle\int_0^\infty \operatorname{\mathbb{E}}\left[ e^{-\lambda X_t Y_u}\right] du
  \end{IEEEeqnarray}
  where all the stochastic processes are independent except the final acceleration $X_v$
  and the increment $Y_v$.

  The Gompertz process introduces an irreducible exponential dependence on age
  \emph{``older organisms are red-shifted with respect to younger organisms''} that cannot be
  removed or linearized by a coordinate transform, analogous to the Hubble constant which
  describes the intrinsic expansion of space-time and for which no coordinate transform can
  remove the intrinsic red-shift of space-time. We can concretely predict that the age of an
  organism introduces a correlation within the latencies between consecutive events, and
  that this correlation can be fully accounted for by conditioning on the age of the
  organism at the sentinel event. 
  
  Our observations of the Gompertz process have a serious consequence. Every single
  longitudinal experiment that has studied outcomes whose latencies are on the same
  timescale as the lifespan of the investigated organism has introduced spurious
  correlations into their longitudinal analysis of event latency, which are purely an
  artefact of the failing to account for the exponential age dependence of the Gompertz
  process. Fortunately, the majority of the studies of biological systems have either been
  cross-sectional or of short enough duration that the hazard of the Gompertz process is
  approximately constant over the timescale of the experiment on the organism.

  \section{L\'evy}
  Integrated geometric Brownian motion $Y_t$ and its carrier process of geometric Brownian
  motion $X_t$ can be embedded into the Lie group of $2 \times 2$ upper triangular
  matrices $\mathfrak{h}_2$ by means of the factorization observed earlier, so that the
  increments are independent under matrix multiplication, for times $t > s$:
  \begin{IEEEeqnarray}{rCl}
    \begin{bmatrix}
      1 & Y_t\\
      0 & X_t
    \end{bmatrix}
    & = &
    \begin{bmatrix}
      1 & Y_{t-s}\\
      0 & X_{t-s}
    \end{bmatrix}
    \begin{bmatrix}
      1 & Y_s\\
      0 & X_s
    \end{bmatrix}
  \end{IEEEeqnarray}
  where the increment processes of $Y_{t-s}$ and $X_{t-s}$ are independent of the processes
  $Y_s$ and $X_s$. It follows that the infinitesimal generator of the matrix exponential map
  into the group is the sub-algebra of the Lie algebra of upper triangular matrices
  $\mathfrak{h}_2$ consisting of first column zero matrices:
  \begin{IEEEeqnarray}{rCl}
    \exp\left(
      \frac{\mu t + \sigma W_t}{X_t - 1}
      \begin{bmatrix}
        0 & Y_t\\
        0 & X_t - 1
      \end{bmatrix}
    \right)
    & = &
    \begin{bmatrix}
      1 & Y_t\\
      0 & X_t
    \end{bmatrix}
  \end{IEEEeqnarray}
  As we will see next this raises the prospect of a deep connection between the adjoint
  derivative of the exponential map and stochastic differential equations.

  \section{Fokker-Planck}
  The upper triangular L\'evy process of integrated geometric Brownian motion satisfies the
  stochastic differential equation:
  \begin{IEEEeqnarray}{rCl}
    d \begin{bmatrix}
      1 & Y_t\\
      0 & X_t
    \end{bmatrix}
    & = &
    \begin{bmatrix}
      0 & X_t\\
      0 & \left(\mu + \sigma^2 / 2\right) X_t
    \end{bmatrix} dt
    +
    \begin{bmatrix}
      0 & 0\\
      0 & \sigma X_t
    \end{bmatrix}dW_t\\\nonumber\\
    & = &
    \left(
      \begin{bmatrix}
        0 & 1\\
        0 & \mu + \sigma^2 / 2
      \end{bmatrix} dt
      +
      \begin{bmatrix}
        0 & 0\\
        0 & \sigma
      \end{bmatrix}dW_t
    \right)
    \begin{bmatrix}
      1 & Y_t\\
      0 & X_t
    \end{bmatrix}
  \end{IEEEeqnarray}
  It follows from Fokker-Planck that the probability density of the joint process 
  $p=\operatorname{\mathbb{P}}\left[Y_t=y,X_t=x\right]$ satisfies
  the partial differential equation:
  \begin{IEEEeqnarray}{rCl}
    \frac{\partial p}{\partial t}
    & = &
    - \frac{\partial}{\partial x} \left(\mu + \frac{\sigma^2}{2}\right)xp
    - \frac{\partial}{\partial y} x p
    + \frac{\partial^2}{\partial x^2} \frac{\sigma^2}{2}x^2 p
  \end{IEEEeqnarray}
  We can restate this as an Eigen evolution equation:
  \begin{IEEEeqnarray}{rCl}
    \frac{\partial p}{\partial t}
    + \left(\mu + \frac{3}{2}\sigma^2\right) x \frac{\partial p}{\partial x}
    + x \frac{\partial p}{\partial y}
    - \frac{\sigma^2}{2}x^2 \frac{\partial^2 p}{\partial x^2}
    & = &
    \left(-\mu + \frac{\sigma^2}{2}\right) p
  \end{IEEEeqnarray}
  Marginalizing over the probability of $X_t$ and applying the acceleration lemma
  yields the first order partial differential equation for the distribution of $Y_t$:
  \begin{IEEEeqnarray}{rCl}
    \frac{\operatorname{\mathbb{E}}\left[ Y_t \right]}
    {\operatorname{\mathbb{E}}\left[ X_t \right]}
    \frac{\partial p}{\partial t}
    - y\frac{\partial p}{\partial y}
    & = &
    p
  \end{IEEEeqnarray}
  Which by trial solution of separation of variables has the general solution:
  \begin{IEEEeqnarray}{rCl}
    p
    & = &
    \operatorname{\mathbb{E}}\left[ Y_t \right]
    f_{\mu, \sigma}\left(y \operatorname{\mathbb{E}}\left[ Y_t \right]\right)
  \end{IEEEeqnarray}
  for any analytic $f_{\mu, \sigma}$ dependent on the drift and diffusion of the carrier
  process $X_t$. Note that we are offloading the dimensional analysis into the analytic
  function.

  \section{Hazard Rate}
  Consider the first passage stopping time $T_1$ of the Gompertz process $G_t$, its tail
  distribution is the characteristic function of $Y_t$:
  \begin{IEEEeqnarray}{rCl}
    \operatorname{\mathbb{P}}\left[ T_1 \ge t\right]
    & = &
    \operatorname{\mathbb{E}}\left[ e^{-\lambda Y_t} \right]
  \end{IEEEeqnarray}
  Thus the hazard rate $h$ of $T_1$:
  \begin{IEEEeqnarray}{rCl}
    h
    & = &
    \operatorname{\mathbb{P}}\left[ T_1 = t\right\rVert\left. T_1 \ge t\right]
  \end{IEEEeqnarray}
  satisfies the partial differential Eigen equation:
  \begin{IEEEeqnarray}{rCl}
    \frac{\operatorname{\mathbb{E}}\left[Y_t\right]}{\operatorname{\mathbb{E}}\left[X_t\right]}
    \frac{\partial h}{\partial t} - \lambda\frac{\partial h}{\partial \lambda}
    & = &
    \left(\frac{\operatorname{\mathbb{E}}\left[Y_t\right]}{\operatorname{\mathbb{E}}\left[X_t\right]}\right)^2
    \left(\frac{\partial}{\partial t}\frac{\operatorname{\mathbb{E}}\left[X_t\right]}{\operatorname{\mathbb{E}}\left[Y_t\right]}\right) h
  \end{IEEEeqnarray}
  Which by trial solution of separation of variables has the general solution:
  \begin{IEEEeqnarray}{rCl}
    h
    & = &
    \lambda^2 \operatorname{\mathbb{E}}\left[X_t\right]
    \operatorname{\mathbb{E}}\left[Y_t\right]
    g_{\mu, \sigma}\left( \lambda \operatorname{\mathbb{E}}\left[Y_t\right] \right)
  \end{IEEEeqnarray}
  for any analytic $g_{\mu, \sigma}$ dependent on the drift and diffusion of
  the carrier process $X_t$. Taking the limit to deterministic subordination yields the
  constraints on $g_{\mu, \sigma}$:
  \begin{longtable}{lll}
    \caption{Sequential Boundary Conditions}\\
    \multicolumn{1}{l}{Boundary} & \multicolumn{1}{l}{Condition} & \multicolumn{1}{l}{Removes}\\
    \hline
    \endfirsthead
    \caption*{Continued from previous page.}\\
    \multicolumn{1}{l}{Boundary} & \multicolumn{1}{l}{Condition} & \multicolumn{1}{l}{Removes}\\
    \hline
    \endhead
    \caption*{Continued on next page.}
    \endfoot
    \caption*{Sequential boundary conditions on the hazard rate $h$ derived from the limits to deterministic subordination.}
    \endlastfoot\\
    $\sigma=0$    & $h=\lambda e^{\mu t}$ & diffusion\\\\
    $\mu=0$       & $h=\lambda$           & then drift\\\\
    $\lambda = 0$ & $h=0$                 & finally jumps
  \end{longtable}
  From the boundary conditions we can immediately deduce that in the deterministic limit of
  $\sigma \rightarrow 0$ we have:
  \begin{IEEEeqnarray}{rCl}
    g_{\mu, \sigma}\left(x\right)
    & \xrightarrow{\sigma = 0} &
    \frac{1}{x}
  \end{IEEEeqnarray}
  However this alone cannot be the solution as it results in the characteristic function in
  $\lambda$ of a purely deterministic $Y_t$. Equating the general solution for the hazard
  rate to the Laplace of the Fokker-Planck solution yields the implicit equation in
  $f_{\mu,\sigma}$ and $g_{\mu,\sigma}$:
  \begin{IEEEeqnarray}{rCl}
    \displaystyle-\frac{\partial}{\partial t} \ln
    \int_0^\infty
    e^\frac{\lambda u}{\operatorname{\mathbb{E}}\left[Y_t\right]}
    f_{\mu,\sigma}\left(u\right)
    du
    & = &
    \lambda^2
    \operatorname{\mathbb{E}}\left[X_t\right]
    \operatorname{\mathbb{E}}\left[Y_t\right]
    g_{\mu, \sigma}\left( \lambda \operatorname{\mathbb{E}}\left[Y_t\right] \right)
  \end{IEEEeqnarray} 
  Dimensional analysis provides an inference for a solution, which remains an open problem:
  \begin{proposition}[Gompertz Anomaly]
    The analytic function $g_{\mu, \sigma}$ is simply the exponential divided by its
    argument, so that the hazard rate $h$ is given by:
    \begin{IEEEeqnarray}{rCl}
      h
      & = &
      \lambda \operatorname{\mathbb{E}}\left[X_t\right]
      e^{\frac{\lambda \sigma^2 / 2}{\mu + \sigma^2 / 2}\operatorname{\mathbb{E}}\left[Y_t\right]}
    \end{IEEEeqnarray}
  \end{proposition}
  \begin{proof}
    The parity multiplied derivatives of the exponential of the integral of the hazard rate
    satisfies the recursion-convolution lemma, generating the moments of $Y_t$, and hence is
    the characteristic function of $Y_t$.
  \end{proof}

  \section{Discussion}
  In most circumstances $\lambda$ is the new born infant mortality due to ageing alone, and
  is less than $1$ in $32000$ person-years. As such the hazard rate is very close to the
  original hazard rate observed by Gompertz. Intuitively, when $\mu \ggg \sigma^2 / 2$ the
  process becomes approximately deterministic due to the large impact of the drift.

  Conversely, the central conjecture in the preliminary material is that ageing is driven by
  stochastic accelerations, requiring that the drift vanish $\mu = 0$. In this case the
  anomalous Gompertz hazard rate simplifies to:
  \begin{IEEEeqnarray}{rCl}
    h
    & = &
    \lambda e^{t\sigma^2 / 2}
    e^{\lambda \frac{e^{t\sigma^2 / 2} - 1}{\sigma^2 / 2}}
  \end{IEEEeqnarray}
  We have observed the anomalous Gompertz hazard rate in mortality rates in Alberta while
  conducting the Hazard Rate Zoo experiment. Specifically when we remove the dominate
  exponential process of a doubling of mortality every $7$ years from an infant mortality of
  $1$ in $32000$ person-years:
  \begin{IEEEeqnarray}{rCl}
    h
    & = &
    \displaystyle\frac{2^{t/7}}{32000}
    e^\frac{7 \left(2^{t/7} - 1\right)}{32000 \ln 2}
  \end{IEEEeqnarray}
  there remains a residual anomalous growth in mortality with ageing, reflecting the higher
  order affect of the stochastic accelerations.
\end{document}