\documentclass{article}

% External references
\usepackage{amsmath}
\usepackage{amssymb}
\usepackage{amsthm}
\usepackage{IEEEtrantools}
\usepackage{mathrsfs}
\usepackage{longtable}

% Mathematical elements
\newtheorem{theorem}{Theorem}
\newtheorem{corollary}{Corollary}
\newtheorem{lemma}{Lemma}
\newtheorem{observation}{Observation}
\newtheorem{proposition}{Proposition}
\theoremstyle{definition}\newtheorem{definition}{Definition}
\renewcommand{\IEEEproofindentspace}{0em}
\renewcommand{\IEEEQED}{\IEEEQEDopen}

\begin{document}
  \title{Gompertz Processes}
  \author{Aaron Geoffrey Sheldon}
  \date{\today}
  \maketitle

  \begin{abstract}
    Motivated by considering infinitesimal stochastic accelerations of time, we outline a
    theory of Gompertz processes, Poisson processes subordinated by integrated Geometric
    Brownian motion.
  \end{abstract}

  \section{Preliminaries}
  Basic science experiments in biology have observed that organisms respond to environmental
  stress, including communicable diseases and exposures to toxins, with an acceleration in
  their failure time. Outside of the controlled setting of a laboratory the environmental
  stresses occur stochastically, resulting in a sequence of accelerations. The accelerations
  can either increase the rate of failure, an exacerbation, or decrease the rate of failure,
  a recovery. Even in the controlled setting of a laboratory it is very difficult to measure
  the underlying metabolic time that is being accelerated, instead we have access to the
  failure time in bare time units.

  Integrated geometric Brownian motion occurs in the asymptotic limit of infinitesimal
  stochastic accelerations. By L\'evy-Khintchine this is the only representation that is
  L\'evy and jump free.

  \begin{definition}[Gompertz Process]
    A Gompertz process $G_t$ is a subordinated Poisson process $N_t$, with rate $\lambda$,
    where the subordinating process is integrated geometric Brownian motion $Y_t$, with
    drift $\mu$ and diffusion $\sigma$:
    \begin{IEEEeqnarray}{rCl}
      G_t
      & = & 
      N_{Y_t}
    \end{IEEEeqnarray}
    given:
    \begin{IEEEeqnarray}{rCl}
      Y_t
      & = & 
      \int_0^t X_s ds\\
      & = &
      \int_0^t e^{\mu s + \sigma W_s} ds
    \end{IEEEeqnarray}
  \end{definition}
  We will make great use of the increment relationship between $Y_t$ and its carrier process
  $X_t$, for times $t > s$:
  \begin{IEEEeqnarray}{rCl}
    Y_t - Y_s
    & \sim &
    X_s Y_{t-s}
  \end{IEEEeqnarray}
  where the process $X_s$ is independent of the process $Y_{t-s}$.

  \begin{lemma}[Acceleration Lemma]
    The expectation of non-negative integer powers $n \ge 0 $ of the carrier process of
    geometric Brownian motion $X_t$ conditioned on integrated geometric Brownian motion
    $Y_t$ is given by:
    \begin{IEEEeqnarray}{rCl}
      \operatorname{\mathbb{E}}\left[ X_t^n \right\rVert\left. Y_t \right]
      & = &
      \frac{\operatorname{\mathbb{E}}\left[X_t^n\right]}{\operatorname{\mathbb{E}}\left[Y_t^n\right]}Y^n_t
    \end{IEEEeqnarray}
  \end{lemma}

  \begin{lemma}[Recursion Convolution Lemma]
    The expectation of non-negative integer powers $n \ge 0 $ of integrated geometric
    Brownian motion $Y_t$ is given by:
    \begin{IEEEeqnarray}{rCl}
      \operatorname{\mathbb{E}}\left[ Y_t^n \right]
      & = &
      n \int_0^t e^{\left(n\mu + \frac{n^2}{2} \sigma^2\right)u} \operatorname{\mathbb{E}}\left[ Y_{t-u}^{n-1} \right] du
    \end{IEEEeqnarray}
  \end{lemma}

  \section{Martingale}
  The scaled two point increments of integrated geometric Brownian motion $Y_t$ form a
  martingale, for a triplet of times $t_1 > t_0 > t_{-1}$:
  \begin{IEEEeqnarray}{rCl}
    \IEEEeqnarraymulticol{3}{l}
    {
      \operatorname{\mathbb{E}}
      \left[
        \frac{Y_{t_1} - Y_{t_0}}{e^{\left(\mu + \frac{\sigma^2}{2}\right)t_1} - e^{\left(\mu + \frac{\sigma^2}{2}\right)t_0}} 
        \right\rVert\left.
        \frac{Y_{t_0} - Y_{t_{-1}}}{e^{\left(\mu + \frac{\sigma^2}{2}\right)t_0} - e^{\left(\mu + \frac{\sigma^2}{2}\right)t_{-1}}} \right]
    }\nonumber\\
    \qquad\qquad\qquad\qquad\qquad\qquad\qquad\qquad
    & = &
    \frac{Y_{t_0} - Y_{t_{-1}}}{e^{\left(\mu + \frac{\sigma^2}{2}\right)t_0} - e^{\left(\mu + \frac{\sigma^2}{2}\right)t_{-1}}}
  \end{IEEEeqnarray}

  \section{L\'evy Process}
  Integrated geometric Brownian motion $Y_t$ and its carrier process of geometric Brownian
  motion $X_t$ can be embedded into the Lie algebra of $2 \times 2$ upper triangular
  matrices $\mathfrak{h}_2$ so that the increments are independent under matrix
  multiplication, for times $t > s$:
  \begin{IEEEeqnarray}{rCl}
    \begin{bmatrix}
      1 & Y_t\\
      0 & X_t
    \end{bmatrix}
    & \sim &
    \begin{bmatrix}
      1 & \tilde{Y}_{t-s}\\
      0 & \tilde{X}_{t-s}
    \end{bmatrix}
    \begin{bmatrix}
      1 & Y_s\\
      0 & X_s
    \end{bmatrix}
  \end{IEEEeqnarray}
  where the increment processes of $\tilde{Y}_{t-s}$ and $\tilde{X}_{t-s}$ are independent
  of the processes $Y_s$ and $X_s$.

  \section{Fokker-Planck}
  The upper triangular L\'evy process of integrated Brownian motion satisfies the stochastic
  differential equation:
  \begin{IEEEeqnarray}{rCl}
    d \begin{bmatrix}
      1 & Y_t\\
      0 & X_t
    \end{bmatrix}
    & = &
    \begin{bmatrix}
      0 & X_t\\
      0 & \left(\mu + \frac{\sigma^2}{2}\right) X_t
    \end{bmatrix} dt
    +
    \begin{bmatrix}
      0 & 0\\
      0 & \sigma X_t
    \end{bmatrix}dW_t
  \end{IEEEeqnarray}
  It follows from Fokker-Planck that the probability density of the joint process 
  $p=\operatorname{\mathbb{P}}\left[Y_t=y,X_t=x\right]$ satisfies
  the partial differential equation:
  \begin{IEEEeqnarray}{rCl}
    \frac{\partial p}{\partial t}
    & = &
    - \frac{\partial}{\partial x} \left(\mu + \frac{\sigma^2}{2}\right)xp
    - \frac{\partial}{\partial y} x p
    + \frac{\partial^2}{\partial x^2} \frac{\sigma^2}{2}x^2 p
  \end{IEEEeqnarray}
  We can restate this as an Eigen evolution equation:
  \begin{IEEEeqnarray}{rCl}
    \frac{\partial p}{\partial t}
    + \left(\mu + \frac{3}{2}\sigma^2\right) x \frac{\partial p}{\partial x}
    + x \frac{\partial p}{\partial y}
    - \frac{\sigma^2}{2}x^2 \frac{\partial^2 p}{\partial x^2}
    & = &
    \left(-\mu + \frac{\sigma^2}{2}\right) p
  \end{IEEEeqnarray}

  \section{Hazard Rate}
  Consider the first passage stopping time $T$ of the Gompertz process $G_t$, its cumulative
  distribution is the characteristic functions of $Y_t$:
  \begin{IEEEeqnarray}{rCl}
    \operatorname{\mathbb{P}}\left[ T \ge t\right]
    & = &
    \operatorname{\mathbb{E}}\left[ e^{-\lambda Y_t} \right]
  \end{IEEEeqnarray}
  Thus the hazard rate $h$ of $T$:
  \begin{IEEEeqnarray}{rCl}
    h
    & = &
    \operatorname{\mathbb{P}}\left[ T = t\right\rVert\left. T \ge t\right]
  \end{IEEEeqnarray}
  satisfies the partial differential Eigen equation:
  \begin{IEEEeqnarray}{rCl}
    \frac{\operatorname{\mathbb{E}}\left[Y_t\right]}{\operatorname{\mathbb{E}}\left[X_t\right]}
    \frac{\partial h}{\partial t} - \lambda\frac{\partial h}{\partial \lambda}
    & = &
    \left(\frac{\operatorname{\mathbb{E}}\left[Y_t\right]}{\operatorname{\mathbb{E}}\left[X_t\right]}\right)^2
    \left(\frac{\partial}{\partial t}\frac{\operatorname{\mathbb{E}}\left[X_t\right]}{\operatorname{\mathbb{E}}\left[Y_t\right]}\right) h
  \end{IEEEeqnarray}
  Which by trial solution of separation of variables has the general solution:
  \begin{IEEEeqnarray}{rCl}
    h & = & \lambda^2 \operatorname{\mathbb{E}}\left[X_t\right] \operatorname{\mathbb{E}}\left[Y_t\right] \operatorname{f}\left( \lambda \operatorname{\mathbb{E}}\left[Y_t\right] \right)
  \end{IEEEeqnarray}
  for any $\operatorname{f}$ analytic.
\end{document}