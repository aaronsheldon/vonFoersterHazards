\documentclass{article}

% External references
\usepackage{amsmath}
\usepackage{amssymb}
\usepackage{amsthm}
\usepackage{mathrsfs}
\usepackage{IEEEtrantools}
\usepackage{mathrsfs}
\usepackage{longtable}

% Mathematical elements
\newtheorem{theorem}{Theorem}
\newtheorem{corollary}{Corollary}
\newtheorem{lemma}{Lemma}
\newtheorem{observation}{Observation}
\newtheorem{proposition}{Proposition}
\theoremstyle{definition}\newtheorem{definition}{Definition}
\renewcommand{\IEEEproofindentspace}{0em}
\renewcommand{\IEEEQED}{\IEEEQEDopen}
\DeclareRobustCommand{\stirling}{\genfrac\{\}{0pt}{}}

% Bibliographic elements
\title{Gompertz Processes: A Theory of Ageing}
\author{Aaron Geoffrey Sheldon}
\date{\today}

\begin{document}
  \maketitle

  \begin{abstract}
    Ageing is process of accelerating failures that is universal to all biological systems.
    Motivated by considering infinitesimal stochastic accelerations of time, I hypothesize
    a general explanation for the emergent determinism of ageing processes in the theory of
    Gompertz processes: Poisson processes subordinated by integrated geometric Brownian
    motion.
  \end{abstract}

  \section{Preliminaries}
  Basic science experiments in biology have ubiquitously observed that organisms respond to
  environmental stresses, including communicable diseases and exposures to toxins, with an
  acceleration $a$ in their failure time $a h\left(a t\right)$, where $h\left(t\right)$ is
  the unperturbed hazard rate of the failure event. Outside of the controlled setting of a
  laboratory, the environmental stresses are random resulting in a stochastic sequence of
  accelerations $a_n$ of the organism's metabolic time at actual times $t_n$. The stochastic
  accelerations can either increase the rate of failure $a_n > 1$, an exacerbation of the
  stresses, or decrease the rate of failure $a_n < 1$, an alleviation from the stresses.
  However, even in the controlled setting of a laboratory it is experimentally challenging
  to directly measure the stochastically accelerated metabolic time of the organism, instead
  we only have access to the failure events measured in bare time units. Thus, a theory of
  ageing must study the subordination of failure times by a stochastically accelerated
  metabolic time.

  Over the course of an organism's lifetime it will encounter exacerbations and alleviations
  that stochastically accelerate $a_n,\dots,a_0=1$ metabolic time at times
  $t_n>\dots>t_0=0$. Furthermore, because the stochastic accelerations are all positive
  $a_i > 0$ for each stochastic acceleration we can find a finite real valued generator
  $w_n,\dots,w_0=0$ such that $a_i = e^{w_i}$. It follows that the stochastically
  accelerated metabolic time $y_{t_n}$ at time $t_n$ is the sum of the products of the
  stochastic accelerations up to time $t_{n-1}$ and the elapsed bare time steps
  $t_i - t_{i-1}$:
  \begin{IEEEeqnarray}{rCl}
    y_{t_n}
    & = & 
    \displaystyle \sum_{i=1}^{n}e^{\sum_{j=0}^{i-1} w_j} \left(t_i - t_{i-1}\right)
  \end{IEEEeqnarray}
  Provided no great explosions of acceleration occur in any small time scale, like say when
  an actual explosion occurs, the generators will become infinitesimal $w_i \rightarrow 0$
  on the same order as the bare time steps become infinitesimal
  $t_i - t_{i-1} \rightarrow 0$:
  \begin{IEEEeqnarray}{rCl}
    \mathcal{O}\left(w_i\right)
    & = & 
    \mathcal{O}\left(t_i - t_{i-1}\right)
  \end{IEEEeqnarray}
  The sum of products then becomes a stochastic process $Y_t$ that is an integral of
  a random geometric infinitesimal generator process $e^{W_t}$:
  \begin{IEEEeqnarray}{rCl}
    Y_t
    & = & 
    \int_0^t e^{W_u} du
  \end{IEEEeqnarray}
  This asymptotic argument is essentially the continuous part of Kolmorgorov's
  characterization of stochastic processes as being composed of, in bounded time, either a
  finite number of discrete jumps or an infinite number of continuous changes.
  
  If we reasonably assume as a first approximation that the exacerbations and alleviations,
  and their respective stochastic accelerations, are independent and stationary over time
  then by the L\'evy-Khintchine characterization the only infinitesimal generator of
  stochastically accelerated metabolic time that is L\'evy and continuous
  \emph{``jump free''} is Brownian motion $W_t$. The stochastically accelerated metabolic
  time $Y_t$ is better known as integrated geometric Brownian motion.

  \section{Gompertz}
  Motivated by the preceding heuristic derivation we formally define the Gompertz process on
  time $t \ge 0$.
  \begin{definition}[Gompertz Process]
    A Gompertz process $G_t$ is a subordinated Poisson process $N_t$, with rate $\lambda$,
    where the subordinating process is integrated geometric Brownian motion $Y_t$, with
    drift $\mu$ and diffusion $\sigma$:
    \begin{IEEEeqnarray}{rCl}
      G_t
      & = & 
      N_{Y_t}
    \end{IEEEeqnarray}
    given:
    \begin{IEEEeqnarray}{rCl}
      Y_t
      & = & 
      \int_0^t X_s ds\\
      & = &
      \int_0^t e^{\mu s + \sigma W_s} ds
    \end{IEEEeqnarray}
  \end{definition}
  Phenomenologically the finite real stochastic process $\mu t + \sigma W_s$ is the
  infinitesimal acceleration at time $t$ that generates a non-negative geometric stochastic
  process $X_t$ of accumulated accelerations up to time $t$ and whose integral $Y_t$ is a
  strictly increasing stochastically accelerated metabolic time up to time $t$.

  From the definition of the Gompertz process $G_t$ conditioning on the history of a sample
  path of integrated geometric Brownian motion $Y_t$ yields a conditional Poisson process,
  where $\operatorname{d\mathbb{P}}$ represents the Radon-Nikodym density:
  \begin{IEEEeqnarray}{rCl}
    \operatorname{d\mathbb{P}}\left[G_t = n \right\rVert\left. Y_t \right]
    & = & 
    \frac{\left(\lambda Y_t\right)^n}{n!} e^{-\lambda Y_t}\\
    \nonumber\\
    \operatorname{\mathbb{E}}\left[G_t^n \right\rVert\left. Y_t \right]
    & = &
    \sum_{k=0}^n \stirling{n}{k} \left(\lambda Y_t\right)^k
  \end{IEEEeqnarray}
  Where the sum in the expectation is over Stirling's numbers of the second kind. Applying
  Bayes' theorem we can condition on the Gompertz process to estimate the elapsed
  stochastically accelerated metabolic time:
  \begin{IEEEeqnarray}{rCl}
    \operatorname{d\mathbb{P}}\left[Y_t \right\rVert\left. G_t =n \right]
    & = &
    \frac{Y_t^n e^{-\lambda Y_t}}
    {\operatorname{\mathbb{E}}\left[Y_t^n e^{-\lambda Y_t} \right]}
    \operatorname{d\mathbb{P}}\left[Y_t \right]\\
    \nonumber\\
    \operatorname{\mathbb{E}}\left[Y_t^m \right\rVert\left. G_t = n \right]
    & = &
    \frac{\operatorname{\mathbb{E}}\left[Y_t^{n+m} e^{-\lambda Y_t} \right]}
    {\operatorname{\mathbb{E}}\left[Y_t^n e^{-\lambda Y_t} \right]}
  \end{IEEEeqnarray}
  Reflecting that integrated geometric Brownian motion $Y_t$ measures the elapsed
  stochastically accelerated metabolic time.

  \section{Factorization}
  To start our exploration of the rich subtleties of the Gompertz process we will briefly
  review the properties of integrated geometric Brownian motion which are salient to
  developing our theory. This is by no means a comprehensive compendium. Much of the
  material I will cover has been deeply and thoroughly explored in the quantitative finance
  literature in the theory of pricing Asian options, and in the graduate syllabus of
  stochastic processes covering subordinated Poisson processes, usually in the context of
  deriving the characteristic function of the subordinating process.
  
  Our first observation is that the increments of $Y_t$ can be factored by its carrier
  process $X_t$, for times $t > s$ we have in law:
  \begin{IEEEeqnarray}{rCl}
    Y_t - Y_s
    & = &
    X_s Y_{t-s}
  \end{IEEEeqnarray}
  where the process $X_s$ is independent of the process $Y_{t-s}$. Phenomenologically the
  increments of $Y_t - Y_s$ are equivalent to a process $Y_{t-s}$ that starts with
  acceleration $X_s$.
  
  Analogous to the Fundamental Theorem of Arithmetic carrier factorization provides for
  immediate shallow inferences. For example we can immediately observe that for times
  $t > s$:
  \begin{IEEEeqnarray}{rCl}
    \operatorname{\mathbb{E}}\left[ \left(Y_t - Y_s\right)^n \right\rVert\left. X_s \right]
    & = &
    X_s^n \operatorname{\mathbb{E}}\left[ Y_{t-s}^n \right]
  \end{IEEEeqnarray}
  I will liberally exploit this technique of arbitraging the stochastically accelerated
  metabolic time $Y_t$ against the accumulated stochastic accelerations $X_t$ to reduce
  expectations down to the well known standard terms for $X_t$ and $Y_t$:
  \begin{IEEEeqnarray}{rCl}
    \operatorname{\mathbb{E}}\left[ X_t^n \right]
    & = &
    e^{\left(n \mu + n^2 \sigma^2 / 2\right)t}\\
    \nonumber\\
    \operatorname{\mathbb{E}}\left[ Y_t \right]
    & = &
    \frac{e^{\left(\mu + \sigma^2 / 2\right)t} - 1}{\mu + \sigma^2 / 2}
  \end{IEEEeqnarray}
  Note that in the last equation I have implicitly invoked the Fubini-Tonelli Theorem to
  switch the order of integration, and will broadly continue to use this theorem throughout
  this work to evaluate expectations, such as the covariance between $Y_t$ and $X_t$:
  \begin{IEEEeqnarray}{rCl}
    \operatorname{\mathbb{C}ov}\left[ X_t, Y_t \right]
    & = &
    \frac{\operatorname{\mathbb{V}ar}\left[ X_t \right]}
    {\mu+3 \sigma^2 / 2}
    - \operatorname{\mathbb{E}}\left[ X_t \right]\operatorname{\mathbb{E}}\left[ Y_t \right]
  \end{IEEEeqnarray}

  Carrier factorization implies that $Y_{t+\Delta t}$ can be recovered through
  marginalization over $Y_t$ and $Y_{\Delta t}$. Applying this observation to the definition
  of the derivative we obtain the hazard rate relationship:
  \begin{IEEEeqnarray}{rCl}
    \frac{d}{dt}
    \operatorname{\mathbb{E}}\left[ X_t \right\rVert\left. Y_t \ge y \right]
    & = &
    \left(\mu + \sigma^2/2\right)
    \operatorname{\mathbb{E}}\left[ X_t \right\rVert\left. Y_t \ge y \right]\nonumber\\
    &   &
    \qquad + \operatorname{\mathbb{E}}\left[ X_t \right\rVert\left. Y_t = y \right]
    \operatorname{d\mathbb{H}}\left[ Y_t = y \right]
  \end{IEEEeqnarray}
  where $\operatorname{d\mathbb{H}}\left[ Y_t = y \right] = \operatorname{d\mathbb{P}}\left[ Y_t = y \right\rVert\left. Y_t \ge y \right]$
  is the hazard rate of the value of $Y_t$. From the integral of this differential equation
  evaluated at $Y_t=0$ we deduce the boundary condition:
  \begin{IEEEeqnarray}{rCl}
    \operatorname{d\mathbb{P}}\left[ X_t = x, Y_t = 0 \right]
    & = &
    \frac{\delta\left(x\right)}{t}
  \end{IEEEeqnarray}

  There are limits to what can be learned from carrier factorization alone. By the
  application of It\^o calculus conditioning on $X_t$ generates a measure space that is the
  product of two $\sigma$-finite measure spaces. We cannot say the same for conditioning on
  $Y_t$. To illustrate the difficulty consider the conditional probability:
  \begin{IEEEeqnarray}{rCl}
    \operatorname{\mathbb{P}}\left[ X_t \right\rVert\left. Y_t \right]
    & = &
    \lim_{n \rightarrow \infty}
    \begin{array}{cl}
      \displaystyle{\frac{1}{n!} \int \cdots \int}
      &
      \operatorname{d\mathbb{P}}\left[ X_\frac{t}{n} = u_1 \right] \cdots\\
      \begin{array}{c}
        \frac{t}{n} \sum\prod u_i = Y_t,
        \\
        \prod u_i = X_t
      \end{array}
      &
      \qquad \cdots \operatorname{d\mathbb{P}}\left[ X_\frac{t}{n} = u_n \right]du_1 \cdots du_n
    \end{array}
  \end{IEEEeqnarray}
  The conditional probability irreducibly mixes the time $s<t$ and accelerations $X_s$, so
  that the measure space of the conditional probability is not the product measure space of
  two $\sigma$-finite measure spaces. As such we cannot apply the Fubini-Tonelli theorem to
  integrate processes conditioned on their integrals. The crux of the failure is that the
  conditioning $X_t$ on it's integral $Y_t$ introduces a reciprocal constraint between the
  size of the excursions of $X_s$ for $s<t$ and the duration of the excursions of $X_s$,
  because by the construction of the integral the area under the excursions of $X_s$ must be
  less than the integral $Y_t$.

  \section{Engelbert-Schmidt}
  At outset of this research project contemplation of the bridged conditioning
  $\operatorname{\mathbb{P}}\left[ X_s \right\rVert\left. Y_t \right]$ of a continuous
  stochastic process $X_s>0$ on it's future integral $Y_t$ for times $s<t$ lead me to have
  deep reservations that the central concepts of my investigations were not well formed.
  Specifically, from the factorization:
  \begin{IEEEeqnarray}{rCl}
    \operatorname{\mathbb{P}}\left[ X_s \le x \right\rVert\left. Y_t \le y \right]
    & = &
    \int_0^x \int_0^y
    \frac{
      \operatorname{d\mathbb{P}}\left[ X_s = u, Y_s = v \right]
      \operatorname{\mathbb{P}}\left[ Y_{t-s} \le \frac{y-v}{u} \right]
    }
    {\operatorname{\mathbb{P}}\left[ Y_t \le y \right]}
    dv du
  \end{IEEEeqnarray}
  we can see that $\operatorname{\mathbb{P}}\left[ X_s \right\rVert\left. Y_t \right]$ being
  well formed hinges on $\operatorname{\mathbb{P}}\left[ X_t , Y_t \right]$ being well
  formed. As such I will undertake a certain amount of measure theoretic \emph{``worrying''}
  and \emph{``hand wringing''} to establish that sets of the joint processes of $X_t$ and
  $Y_t$ are adapted to the filtration $\mathscr{F}_t$ of the carrier Brownian motion $W_t$.
  Largely for my own pedantic edification I will informally derive the central result of the
  Engelbert-Schmidt zero-one law, from elementary first principles; namely that the integral
  of $X_t$ specifies $\mathscr{F}_t$ measurable sets.

  The concern of the Engelbert-Schmidt zero-one law is that probability measures of the
  integral $Y_t$ of a continuous stochastic process $X_t$ are implicitly probability
  measures of the limit of sets of $X_s$ with $s<t$:
  \begin{IEEEeqnarray}{rCl}
    \operatorname{\mathbb{P}}\left[ Y_t, \dots \right]
    & = &
    \operatorname{\mathbb{P}}\left[ Y_t = \lim_{\pi \subset \left[0,t\right)} \sum_{t_i \in \pi} X_{t_i} \Delta t_i, \dots \right]
  \end{IEEEeqnarray}
  The heart of the result is to show that the limit can be countably constructed from
  measurable sets in $\mathscr{F}_s \subset \mathscr{F}_t$, the filtration of $X_t$. As
  such, consider the sets of successively closer Lorentz factor approximations of the lower
  bounded integral $Y_t>y$, with the Riemann sum indexed by integers $n \ge 2$:
  \begin{IEEEeqnarray}{rCl}
    A_n
    & = &
    \left\lbrace
      \sum_{m=0}^{n-2} X_{\frac{n}{m}t} \vee X_{\frac{n+1}{m}t}
      > \left(n-\sqrt{n}\right)\frac{y}{t}
    \right\rbrace
  \end{IEEEeqnarray}
  I have approximated the area using the maximum of consecutive evaluations of $X_t$ to
  ensure all the stragglers of $Y_t$ that just barely cross $y$ near $t$ are captured in the
  sets. To define the joint probability of $X_t > x$ and $Y_t > y$ we constrain the
  approximations along the direction $x$:
  \begin{IEEEeqnarray}{rCl}
    B_n
    & = &
    \left\lbrace
      \sum_{m=0}^{n-2} X_{\frac{n}{m}t} \vee X_{\frac{n+1}{m}t}
      > \left(n\frac{y}{t} - x\right) \vee 0
    \right\rbrace
  \end{IEEEeqnarray}
  The sets push the approximations of $Y_{\frac{n-1}{n}t}$ into the future $Y_t$ so that the
  stopped indicator processes of:
  \begin{IEEEeqnarray}{rCl}
    A_{s \uparrow t}
    & = &
    \begin{cases}
      \operatorname{\mathbb{I}}\left[A_{\left\lceil \frac{t}{t-s} \right\rceil}\right]
      & s < t,\\
      0
      & \text{else}
    \end{cases}\\
    B_{s \uparrow t}
    & = &
    \begin{cases}
      \operatorname{\mathbb{I}}\left[B_{\left\lceil \frac{t}{t-s} \right\rceil}\right]
      & s < t,\\
      0
      & \text{else}
    \end{cases}\\
  \end{IEEEeqnarray}
  are adapted to the filtration $\mathscr{F}_s \subset \mathscr{F}_t$ of the carrier
  Brownian motion $W_s$ for $s<t$. Liberally invoking the continuity of $X_t$ we define
  the integral process $Y_t>y$ by requiring that the approximations eventually hold in the
  limit inferior sense:
  \begin{IEEEeqnarray}{rCl}
    \liminf A_n
    & = &
    \left\lbrace X_t \in A_n \text{ eventually } \forall n \right\rbrace\\
    & = &
    \left\lbrace X_t < \infty, \int_0^t X_u du > y \right\rbrace\\
    \liminf B_n
    & = &
    \left\lbrace X_t \in B_n \text{ eventually } \forall n \right\rbrace\\
    & = &
    \left\lbrace X_t < x, \int_0^t X_u du > y \right\rbrace
  \end{IEEEeqnarray}
  The danger is that the containing limit superior asymptotically induces oscillations in
  $Y_t$ of increasing frequency and decreasing magnitude so that $X_t$ becomes
  \emph{``fuzzy''} at $t$, and thus fails to be a set in the filtration $\mathscr{F}_t$.
  Essentially we are faced with the risk of a path integral formulation of the Heisenberg
  uncertainty principle. However, the limit superior and limit inferior differ only by
  negligible sets in $\mathscr{F}_t$:
  \begin{IEEEeqnarray}{rCl}
    \limsup A_n - \liminf A_n
    & = &
    \left\lbrace X_t \rightarrow \infty, \int_0^t X_u du = y \right\rbrace\\
    & \subset &
    \left\lbrace  X_t \rightarrow \infty \right\rbrace\\
    \limsup B_n - \liminf B_n
    & = &
    \left\lbrace X_t = x, \int_0^t X_u du = y \right\rbrace\\
    & \subset &
    \left\lbrace X_t = x \right\rbrace
  \end{IEEEeqnarray}
  Although the limit inferiors are strict subsets of the limit superiors they are
  nonetheless equivalent in probability in $\mathscr{F}_t$, and all the undefined
  \emph{``fuzziness''} of $X_t$ induced by the limit superior is pruned out by the
  restriction of the filtration to continuous processes.

  The difference between the two limit sets characterizes the exceedence functions of the
  approximations, and can be interpreted as the directional derivatives of the marginal and
  joint probabilities along $Y_t=y$ and $X_t=x, Y_t=y$ respectively. The difference sets
  have negligible measure because we are implicitly working within the integral of the
  probability measure and not the Radon-Nikodym probability density.
  
  For clarity I have used Riemann sums. If one were to tackle this argument rigorously they
  would be required to work with a countable net of finite partitions of time. Nonetheless
  through the countable construction of the limit inferiors we reach the future constraints
  by carefully pruning the paths in the of present $\mathscr{F}_s \subset \mathscr{F}_t$,
  without ever actually peaking into the future of $\mathscr{F}_t$ to condition on a future
  event. Thus the marginal probability $\operatorname{\mathbb{P}}\left[Y_t \right]$ and
  joint probability $\operatorname{\mathbb{P}}\left[ X_t , Y_t \right]$ exist and are
  finite.

  \section{Identities}
  Returning from our digression, even with the factorization observation we are still in
  need of a means of reducing the expectation of the powers $Y_t^n$. A small lemma suffices
  to provide the means of finding powers:

  \begin{lemma}[Recursion-Convolution Lemma]
    The expectation of non-negative integer powers $m, n \ge 0 $ of geometric Brownian 
    motion $X_t$ and integrated geometric Brownian motion $Y_t$ is given by:
    \begin{IEEEeqnarray}{rCl}
      \operatorname{\mathbb{E}}\left[X_t^m Y_t^n \right]
      & = &
      n \int_0^t
      \operatorname{\mathbb{E}}\left[X_{t-u}^{m+n}\right]
      \operatorname{\mathbb{E}}\left[ X_u^m Y_u^{n-1} \right] du
    \end{IEEEeqnarray}
  \end{lemma}
  \begin{proof}
    Expanding the expectation as a multi-variable integral, applying the Fubini-Tonelli
    Theorem, factoring, and a final change of variables we have:
    \begin{IEEEeqnarray}{rCl}
      \operatorname{\mathbb{E}}\left[X_t^m  Y_t^n \right]
      & = &
      \operatorname{\mathbb{E}}\left[ \int_0^t \dots \int_0^t X_t^m X_{u_1} \cdots X_{u_n} du_1 \dots du_n \right]\\
      & = &
      \binom{n}{1} \operatorname{\mathbb{E}}\left[ \int_0^t X_t^m X_u \left(Y_t - Y_u\right)^{n-1} du \right]\\
      & = &
      n \int_0^t
      \operatorname{\mathbb{E}}\left[X_{t-u}^{m+n}\right]
      \operatorname{\mathbb{E}}\left[ X_u^m Y_u^{n-1} \right] du
    \end{IEEEeqnarray}
  \end{proof}

  The recursion-convolution lemma relates the moments of integrated geometric Brownian
  motion through a linear operator, and like all good linear operators this relationship
  deserves a uniqueness constraint.

  \begin{corollary}[Uniqueness Corollary]
    If two sequences of functions $f_t^{\left(n\right)}$ and $g_t^{\left(n\right)}$ of time
    $t$ satisfy the recursion-convolution relation and 
    $f_t^{\left(0\right)}=g_t^{\left(0\right)}$ then
    $f_t^{\left(n\right)}=g_t^{\left(n\right)}$ for all $n$.
  \end{corollary}
  \begin{proof}
    We proceed with induction on $n$
    \begin{enumerate}
      \item By assumption for $n=0$ functions are equal.
      \item Now assume that up to $n$ the functions are equal.
      \item Taking the difference between the functions at $n+1$ we have
      \begin{IEEEeqnarray}{rCl}
        f_t^{\left(n+1\right)} - g_t^{\left(n+1\right)}
        & = &
        \displaystyle\int_0^t
        \operatorname{\mathbb{E}}\left[X_{t-u}^{m+n}\right]
        \left(f_u^{\left(n\right)} - g_u^{\left(n\right)}\right) du\\
        & = & 0
      \end{IEEEeqnarray}
    \end{enumerate}
  \end{proof}

  Convolution equations are dual to differential equations, and as such the products of
  powers of $X_t$ and $Y_t$ satisfy the recursive differential equation:
  \begin{IEEEeqnarray}{rCl}
    \frac{d}{dt}\operatorname{\mathbb{E}}\left[ X_t^m Y_t^n \right]
    & = &
    n
    \operatorname{\mathbb{E}}\left[ X_t^m Y_t^{n-1} \right]\nonumber\\
    &   &
    \qquad + \left(\left(m+n\right)\mu + \left(m+n\right)^2\sigma^2 /2 \right)
    \operatorname{\mathbb{E}}\left[ X_t^m Y_t^n \right]
  \end{IEEEeqnarray}

  With the recursion-convolution lemma in hand we have the sufficient tools required to
  estimate all the usual statistics involving powers of $Y_t$, including the expectation,
  variance, and covariances. For example a straightforward, if rather tedious integration
  gives the expected value of the square of integrated geometric Brownian motion:
  \begin{IEEEeqnarray}{rCl}
    \operatorname{\mathbb{E}}\left[ Y_t^2 \right]
    & = &
    \frac{\operatorname{\mathbb{V}ar}\left[ X_t \right]}
    {\left(\mu + \sigma^2\right)\left(\mu + 3\sigma^2 /2 \right)}-
    \frac{\operatorname{\mathbb{E}}\left[ Y_t \right]}
    {\left(\mu + \sigma^2\right)}
  \end{IEEEeqnarray}
  Which by differentiation yields the expectation of the product of $X_t$ and $Y_t$:
  \begin{IEEEeqnarray}{rCl}
    \operatorname{\mathbb{E}}\left[ X_t Y_t \right]
    & = &
    \frac{\operatorname{\mathbb{V}ar}\left[ X_t \right]}
    {\mu + 3\sigma^2 /2 }
  \end{IEEEeqnarray}
  Arriving round trip at a formula that exactly agrees with the previously derived
  covariance.

  \section{Stopping Times}
  In practice we cannot directly measure the stochastically accelerated metabolic time $Y_t$
  instead we have access to the stopping times $T_n$ of the passages $G_{T_n} = n$ of the
  Gompertz process, which we construct in the usual manner. As before, we condition the the
  stopping time on the history of stochastic accelerations to recover the familiar forms of 
  the Poisson process, by differentiating the subordinated cumulative distribution of a
  single stopping time:
  \begin{IEEEeqnarray}{rCl}
    \operatorname{d\mathbb{P}}\left[ T_n = t \right\rVert\left. X_{T_n}, Y_{T_n} \right]
    & = &
    \begin{cases}
      \delta\left(t\right)
      & n = 0,\\
      \displaystyle\frac{\lambda^n X_t Y_t^{\left(n-1\right)}e^{-\lambda Y_t}}
      {\left(n-1\right)!}
      & \text{otherwise}
    \end{cases}
  \end{IEEEeqnarray}
  Applying Bayes' Theorem we can condition on the stopping times to estimate the underlying
  stochastically accelerated metabolic time:
  \begin{IEEEeqnarray}{rCl}
    \operatorname{d\mathbb{P}}\left[ X_{T_n}, Y_{T_n} \right\rVert\left. T_n = t \right]
    & = &
    \begin{cases}
      \delta\left(X_t\right)\delta\left(Y_t\right)
      & n = 0,\\
      \displaystyle\frac{X_t Y_t^{n-1} e^{-\lambda Y_t}}
      {\operatorname{\mathbb{E}}\left[ X_t Y_t^{n-1} e^{-\lambda Y_t} \right]}
      \operatorname{d\mathbb{P}}\left[ X_t,Y_t \right]
      & \text{otherwise}
    \end{cases}
  \end{IEEEeqnarray}
  From this re-weighting of the probabilities we can immediately deduce the elegant
  expectations conditioned on the stopping time:
  \begin{IEEEeqnarray}{rCl}
    \operatorname{\mathbb{E}}\left[ Y_{T_n}^m \right\rVert\left. T_n = t \right]
    & = &
    \begin{cases}
      0
      & n=0,\\
      \displaystyle\frac{\left(n+m-1\right)!}{\lambda^m \left(n-1\right)!}
      \frac{ \operatorname{d\mathbb{P}}\left[ T_{n+m} = t \right]}
      { \operatorname{d\mathbb{P}}\left[ T_n=t \right]}
      & \text{otherwise}
    \end{cases}\\
    \nonumber\\
    \operatorname{\mathbb{E}}\left[ Y_{T_n}^m \right]
    & = &
    \begin{cases}
      0
      & n=0,\\
      \displaystyle\frac{\left(n+m-1\right)!}{\lambda^m \left(n-1\right)!}
      & \text{otherwise}
    \end{cases}\\
    \nonumber\\
    \operatorname{\mathbb{E}}\left[ X_{T_n}^m \right\rVert\left. T_n = t \right]
    & = &
    \begin{cases}
      0
      & n=0,\\
      \displaystyle\frac{\operatorname{\mathbb{E}}\left[ X_t^{m+1} Y_t^{n-1} e^{-\lambda Y_t} \right]}
      {\operatorname{\mathbb{E}}\left[ X_t Y_t^{n-1} e^{-\lambda Y_t} \right]}
      & \text{otherwise}
    \end{cases}
  \end{IEEEeqnarray}
  Finding a closed form for the distribution of the stopping times of the passages of the
  Gompertz process will require a deeper understanding of the characteristic function of
  integrated geometric Brownian motion, which we will develop later. In the meantime there
  are many fruits to be plucked from a study of stopping times of the passages of the
  Gompertz process.
 
  The central statistic of study in the longitudinal analysis of biological systems is the
  latency $T_{1+G_t} - T_{G_t}$ between consecutive stopping times of the passages of the
  Gompertz process. We have subordinated the stopping times $T_{n+G_t}$ of the passages of
  the Gompertz process by the increments of the Gompertz process $n+G_t$ from a sentinel
  event $G_t$ due to immortal time bias \emph{``we cannot see indefinitely into the past''}.
  In practice observational studies, particularly in clinical research and epidemiology, are
  only able to observe consecutive passages of the Gompertz process from a fixed sentinel
  event without the knowledge of how many events have occurred before the sentinel event.
   
  From the preceding probability density we can immediately deduce the tail probability and
  hence the expectation of the latency $s>0$ conditioned on a sentinel event at time
  $t \ge 0$:
  \begin{IEEEeqnarray}{rCl}
    \IEEEeqnarraymulticol{3}{l}{
      \operatorname{\mathbb{P}}\left[ T_{1+G_t} - T_{G_t} \ge s \right\rVert\left. T_{G_t}=t \right]
    } \nonumber\\
    \qquad & = &
    \begin{cases}
      \operatorname{\mathbb{E}}\left[e^{-\lambda Y_s } \right]
      & t = 0,\\
      \displaystyle\sum_{n=1}^\infty \frac{\lambda^n}{n!}
      \frac{\operatorname{\mathbb{E}}\left[X_t Y_t^{n-1} e^{-\lambda Y_{t+s} } \right]}
      {\operatorname{\mathbb{E}}\left[ X_t Y_t^{n-1} e^{-\lambda Y_t} \right]}
      \frac{\operatorname{\mathbb{E}}\left[ Y_t^n e^{-\lambda Y_t} \right]}
      {1 - \operatorname{\mathbb{E}}\left[e^{-\lambda Y_t} \right]}
      & \text{otherwise}
    \end{cases}\\
    \nonumber\\
    \IEEEeqnarraymulticol{3}{l}{
      \operatorname{\mathbb{E}}\left[ T_{1+G_t} - T_{G_t} \right\rVert\left. T_{G_t}=t \right]
    } \nonumber\\
    \qquad & = &
    \begin{cases}
      \operatorname{\mathbb{E}}\left[\int_0^\infty e^{-\lambda Y_u } du \right]
      & t = 0,\\
      \displaystyle\sum_{n=1}^\infty \frac{\lambda^n}{n!}
      \frac{\operatorname{\mathbb{E}}\left[X_t Y_t^{n-1} \int_0^\infty e^{-\lambda Y_{t+u} } du \right]}
      {\operatorname{\mathbb{E}}\left[ X_t Y_t^{n-1} e^{-\lambda Y_t} \right]}
      \frac{\operatorname{\mathbb{E}}\left[ Y_t^n e^{-\lambda Y_t} \right]}
      {1 - \operatorname{\mathbb{E}}\left[e^{-\lambda Y_t} \right]}
      & \text{otherwise}
    \end{cases}
  \end{IEEEeqnarray}
  Clearly if we have available to us the additional information of the full history of event
  counts $G_t$ this difficult expectation becomes much easier as we can elide the
  marginalization over all cardinalities of events.

  Developing this line of reasoning further, consider the covariance of the latency between
  consecutive events $T_{2+G_t},T_{1+G_t}, T_{G_t}$ conditioned on the sentinel event
  $T_{G_t}=t$:
  \begin{IEEEeqnarray}{rCl}
    \IEEEeqnarraymulticol{3}{l}{
      \operatorname{\mathbb{C}ov}\left[ T_{2+G_t} - T_{1+G_t}, T_{1+G_t} - T_{G_t} \right\rVert\left. T_{G_t}=t \right]
    }\nonumber\\
    \qquad
    & = &
    \displaystyle\int_0^\infty\displaystyle\int_0^\infty
    \operatorname{\mathbb{E}}\left[ e^{-\lambda X_{t+v} Y_u}\right]
    \operatorname{\mathbb{E}}\left[\lambda v X_v e^{-\lambda X_t Y_v}\right]
    dv du\nonumber\\
    &   &
    \qquad- \displaystyle\int_0^\infty \operatorname{\mathbb{E}}\left[\lambda X_t Y_u e^{-\lambda X_t Y_u}\right] du
    \displaystyle\int_0^\infty \operatorname{\mathbb{E}}\left[ e^{-\lambda X_t Y_u}\right] du
  \end{IEEEeqnarray}
  where all the stochastic processes are independent except the final acceleration $X_v$
  and the increment $Y_v$.

  The Gompertz process introduces an irreducible exponential dependence on age
  \emph{``older organisms are red-shifted with respect to younger organisms''} that cannot be
  removed or linearized by a coordinate transform, analogous to the Hubble constant which
  describes the intrinsic expansion of space-time and for which no coordinate transform can
  remove the intrinsic red-shift of space-time. We can concretely predict that the age of an
  organism introduces a correlation within the latencies between consecutive events, and
  that this correlation can be fully accounted for by conditioning on the age of the
  organism at the sentinel event. 
  
  Our observations of the Gompertz process have a serious consequence. Every single
  longitudinal experiment that has studied outcomes whose latencies are on the same
  timescale as the lifespan of the investigated organism has introduced spurious
  correlations into their longitudinal analysis of event latency, which are purely an
  artefact of the failing to account for the exponential age dependence of the Gompertz
  process. Fortunately, the majority of the studies of biological systems have either been
  cross-sectional or of short enough duration that the hazard of the Gompertz process is
  approximately constant over the timescale of the experiment on the organism.

  \section{Markov}
  Integrated geometric Brownian motion $Y_t$ is a 2 step Markov process, and thus the
  increments $Y_t - Y_s$, with $t > s$ are Markov. To verify this consider the sequence of
  bare times $0=t_0 < \dots < t_{n+1}=t$, and accelerated times $0=y_0 < \dots < y_{n+1}=y$,
  working through the conditional probability we have:
  \begin{IEEEeqnarray}{rCl}
    \IEEEeqnarraymulticol{3}{l}{
      \operatorname{\mathbb{P}}\left[ Y_{t_{n+1}} - Y_{t_n} \ge y_{n+1} - y_n \right\rVert\left. Y_{t_n} = y_n, \dots , Y_{t_0} = y_0 \right]
    }\nonumber\\
    & \qquad = &
    \operatorname{\mathbb{P}}\left[X_{t_n-t_{n-1}} Y_{t_{n+1}-t_n} \ge \frac{y_{n+1}-y_n}{y_n-y_{n-1}} Y_{t_n-t_{n-1}}\right]\\
    & \qquad = &
    \operatorname{\mathbb{P}}\left[ Y_{t_{n+1}} - Y_{t_n} \ge y_{n+1} - y_n \right\rVert\left.  Y_{t_n} - Y_{t_{n-1}} = y_n - y_{n-1} \right]
  \end{IEEEeqnarray}
  for process $Y_{t_{n+1}-t_n}$ independent of the processes $X_{t_n-t_{n-1}}$ and
  $Y_{t_n-t_{n-1}}$.

  Remarkably the stopping times of the passages of the Gompertz process are Markov, even
  though the stochastically accelerated metabolic time of integrated geometric Brownian
  motion is only Markov in its increments. Specifically knowledge of the stopping $T_n$ is
  sufficient to determine the distribution of the latency $t > 0$ to next event $T_{n+1}$.
  To see this consider the stopping times $T_n=t_n, \dots , T_0 = t_0$, the cumulative
  probability of the latency conditioned of the previous events is:
  \begin{IEEEeqnarray}{rCl}
    \IEEEeqnarraymulticol{3}{l}{
      \operatorname{\mathbb{P}}\left[ T_{n+1} - T_n \ge t \right\rVert\left. T_n=t_n, \dots , T_0 = t_0 \right]
    }\nonumber\\
    \qquad
    & = &
    \displaystyle\sum_{k=0}^\infty \frac{\left(-\lambda\right)^k}{k!}
    \operatorname{\mathbb{E}}\left[Y_s^k\right]
    \operatorname{\mathbb{E}}\left[X_{t_n}^k \right\rVert\left. T_n=t_n, \dots , T_0 = t_0 \right]\\
    \qquad
    & = &
    \frac{\operatorname{\mathbb{E}}\left[X_{t_n} Y_{t_n}^{n-1} e^{-\lambda Y_{t + t_n}} \right]}
    {\operatorname{\mathbb{E}}\left[X_{t_n} Y_{t_n}^{n-1} e^{-\lambda Y_{t_n}} \right]}\\
    \qquad
    & = &
    \operatorname{\mathbb{P}}\left[ T_{n+1} - T_n \ge t \right\rVert\left. T_n=t_n \right]
  \end{IEEEeqnarray}

  \section{L\'evy}
  Integrated geometric Brownian motion $Y_t$ and its carrier process of geometric Brownian
  motion $X_t$ can be embedded into the Lie group of $2 \times 2$ upper triangular
  matrices $H_2$ by means of the factorization observed earlier, so that the
  increments are independent under matrix multiplication, for times $t > s$ we have in law:
  \begin{IEEEeqnarray}{rCl}
    \begin{bmatrix}
      1 & Y_t\\
      0 & X_t
    \end{bmatrix}
    & = &
    \begin{bmatrix}
      1 & Y_{t-s}\\
      0 & X_{t-s}
    \end{bmatrix}
    \begin{bmatrix}
      1 & Y_s\\
      0 & X_s
    \end{bmatrix}
  \end{IEEEeqnarray}
  where the increment processes of $Y_{t-s}$ and $X_{t-s}$ are independent of the processes
  $Y_s$ and $X_s$. It follows that the infinitesimal generator of the matrix exponential map
  into the group is the sub-algebra of the Lie algebra of upper triangular matrices
  $\mathfrak{h}_2$ consisting of first column zero matrices. For example, one branch of the
  logarithm is given by:
  \begin{IEEEeqnarray}{rCl}
    \exp\left(
      \frac{\mu t + \sigma W_t}{X_t - 1}
      \begin{bmatrix}
        0 & Y_t\\
        0 & X_t - 1
      \end{bmatrix}
    \right)
    & = &
    \begin{bmatrix}
      1 & Y_t\\
      0 & X_t
    \end{bmatrix}
  \end{IEEEeqnarray}

  In the next section we will find use for the drift free auxillary stochastic process
  $U_t$; which we motivate by considering the expectation of $Y_t$:
  \begin{IEEEeqnarray}{rCl}
    U_t
    & = &
    Y_t - \frac{X_t - 1}{\mu + \sigma^2 / 2}
  \end{IEEEeqnarray}
  Conveniently the joint process of $X_t$ and $U_t$ enjoys the same embedding into the Lie
  algebra $H_2$ as the original joint process, for times $t > s$ we have in law:
  \begin{IEEEeqnarray}{rCl}
    \begin{bmatrix}
      1 & U_t\\
      0 & X_t
    \end{bmatrix}
    & = &
    \begin{bmatrix}
      1 & U_{t-s}\\
      0 & X_{t-s}
    \end{bmatrix}
    \begin{bmatrix}
      1 & U_s\\
      0 & X_s
    \end{bmatrix}
  \end{IEEEeqnarray}

  A word of caution about the generators in $\mathfrak{h}_2$, because the basis elements of
  the Lie sub-algebra do not commute:
  \begin{IEEEeqnarray}{rCl}
    \left[
      \begin{bmatrix}
        0 & 1\\
        0 & 0
      \end{bmatrix}_,
      \begin{bmatrix}
        0 & 0\\
        0 & 1
      \end{bmatrix}
    \right]
    & = &
    \begin{bmatrix}
      0 & 1\\
      0 & 0
    \end{bmatrix}
  \end{IEEEeqnarray}
  the generator of the product of two increments in $H_2$ is not the sum of the generators
  of each increment. The representation of multiplicative L\'evy processes in Lie groups
  raises the prospect of a deep connection between the adjoint derivative of the exponential
  map and stochastic differential equations.

  \section{Fokker-Planck}
  The upper triangular L\'evy process of integrated geometric Brownian motion satisfies the
  stochastic differential equation:
  \begin{IEEEeqnarray}{rCl}
    d \begin{bmatrix}
      1 & Y_t\\
      0 & X_t
    \end{bmatrix}
    & = &
    \begin{bmatrix}
      0 & X_t\\
      0 & \left(\mu + \sigma^2 / 2\right) X_t
    \end{bmatrix} dt
    +
    \begin{bmatrix}
      0 & 0\\
      0 & \sigma X_t
    \end{bmatrix}dW_t\\
    \nonumber\\
    & = &
    \left(
      \begin{bmatrix}
        0 & 1\\
        0 & \mu + \sigma^2 / 2
      \end{bmatrix} dt
      +
      \begin{bmatrix}
        0 & 0\\
        0 & \sigma
      \end{bmatrix}dW_t
    \right)
    \begin{bmatrix}
      1 & Y_t\\
      0 & X_t
    \end{bmatrix}
  \end{IEEEeqnarray}
  It follows from Fokker-Planck that the Radon-Nikodym probability density of the joint
  process $p=\operatorname{d\mathbb{P}}\left[Y_t=y,X_t=x\right]$ satisfies the partial
  differential equation:
  \begin{IEEEeqnarray}{rCl}
    \frac{\partial p}{\partial t}
    & = &
    - \frac{\partial}{\partial y} x p
    - \frac{\partial}{\partial x} \left(\mu + \frac{\sigma^2}{2}\right)xp
    + \frac{\partial^2}{\partial x^2} \frac{\sigma^2}{2}x^2 p
  \end{IEEEeqnarray}
  We can restate this as an Eigen evolution equation:
  \begin{IEEEeqnarray}{rCl}
    \frac{\partial p}{\partial t}
    + x \frac{\partial p}{\partial y}
    + \left(\mu - \frac{3}{2}\sigma^2\right) x \frac{\partial p}{\partial x}
    - \frac{\sigma^2}{2}x^2 \frac{\partial^2 p}{\partial x^2}
    & = &
    \left(-\mu + \frac{\sigma^2}{2}\right) p
  \end{IEEEeqnarray}
  Rather than directly integrating this equation we instead consider the stochastic
  differential equation of the auxillary process:
  \begin{IEEEeqnarray}{rCl}
    d \begin{bmatrix}
      1 & U_t\\
      0 & X_t
    \end{bmatrix}
    & = &
    \begin{bmatrix}
      0 & 0\\
      0 & \left(\mu + \sigma^2 / 2\right) X_t
    \end{bmatrix} dt
    +
    \begin{bmatrix}
      0 & - \frac{\sigma }{\mu + \sigma^2 / 2}X_t\\
      0 & \sigma X_t
    \end{bmatrix}dW_t\\
    \nonumber\\
    & = &
    \left(
      \begin{bmatrix}
        0 & 0\\
        0 & \mu + \sigma^2 / 2
      \end{bmatrix} dt
      +
      \begin{bmatrix}
        0 & - \frac{\sigma}{\mu + \sigma^2 / 2}\\
        0 & \sigma
      \end{bmatrix}dW_t
    \right)
    \begin{bmatrix}
      1 & U_t\\
      0 & X_t
    \end{bmatrix}
  \end{IEEEeqnarray}
  The distribution of the increments can be derived from the factorization of integrated
  geometric Brownian motion $Y_t$ by geometric Brownian motion $X_t$:
  \begin{IEEEeqnarray}{rCl}
    \IEEEeqnarraymulticol{3}{l}{
    \operatorname{\mathbb{P}}\left[ X_t= x, Y_t=y \right\rVert\left. X_{t_0}= x_0, Y_{t_0}=y_0 \right]
    }\nonumber\\
    \qquad
    & = &
    \operatorname{\mathbb{P}}\left[ X_{t-t_0}=\frac{x}{x_0}, Y_{t-t_0}=\frac{y-y_0}{x_0} \right]
  \end{IEEEeqnarray}

  \section{Hazard Rate}
  Consider the first passage stopping time $T_1$ of the Gompertz process $G_t$, its tail
  distribution is the characteristic function of $Y_t$:
  \begin{IEEEeqnarray}{rCl}
    \operatorname{\mathbb{P}}\left[ T_1 \ge t\right]
    & = &
    \operatorname{\mathbb{E}}\left[ e^{-\lambda Y_t} \right]
  \end{IEEEeqnarray}
  and in turn yields the hazard rate of $T_1$:
  \begin{IEEEeqnarray}{rCl}
    \operatorname{d\mathbb{H}}\left[ T_1 = t\right]
    & = &
    \frac{\lambda \operatorname{\mathbb{E}}\left[X_t e^{-\lambda Y_t} \right]}
    {\operatorname{\mathbb{E}}\left[e^{-\lambda Y_t} \right]}
  \end{IEEEeqnarray}
  From the application of change of variables of integration:
  \begin{IEEEeqnarray}{rCl}
    \lambda Y_t
    & = &
    \int_0^{\lambda t} e^{\frac{\mu}{\lambda} s + \frac{\sigma}{\sqrt{\lambda}} W_s} ds 
  \end{IEEEeqnarray}
  we have the following rescaling of the hazard rate:
  \begin{IEEEeqnarray}{rCl}
    \operatorname{d\mathbb{H}}_{\mu, \sigma, \lambda}\left[ T_1 = t\right]
    & = &
    \lambda\operatorname{d\mathbb{H}}_{\frac{\mu}{\lambda}, \frac{\sigma}{\sqrt{\lambda}}, 1}\left[ T_1 = \lambda t\right]
  \end{IEEEeqnarray}
  Which we can take as the definition of the hazard rate for every value of $\lambda$. As
  such we need only determine $\operatorname{\mathbb{E}}\left[ e^{-Y_t} \right]$. Taking the
  limit to deterministic subordination yields the constraints on the hazard rate:
  \begin{longtable}{lll}
    \caption{Sequential Boundary Conditions}\\
    \multicolumn{1}{l}{Boundary} & \multicolumn{1}{l}{Condition} & \multicolumn{1}{l}{Removes}\\
    \hline
    \endfirsthead
    \caption*{Continued from previous page.}\\
    \multicolumn{1}{l}{Boundary} & \multicolumn{1}{l}{Condition} & \multicolumn{1}{l}{Removes}\\
    \hline
    \endhead
    \caption*{Continued on next page.}
    \endfoot
    \caption*{Sequential boundary conditions on the hazard rate $h$ derived from the limits to deterministic subordination.}
    \endlastfoot\\
    $\sigma=0$    & $\operatorname{d\mathbb{H}}=\lambda e^{\mu t}$ & diffusion\\\\
    $\mu=0$       & $\operatorname{d\mathbb{H}}=\lambda$           & then drift\\\\
    $\lambda = 0$ & $\operatorname{d\mathbb{H}}=0$                 & finally jumps
  \end{longtable}
  Dimensional analysis provides an inference for a solution, which remains an open problem:
  \begin{proposition}[Gompertz Anomaly]
    The hazard rate is given by:
    \begin{IEEEeqnarray}{rCl}
      \operatorname{d\mathbb{H}}
      & = &
      \lambda \operatorname{\mathbb{E}}\left[X_t\right]
      e^{\frac{\lambda \sigma^2 / 2}{\mu + \sigma^2 / 2}\operatorname{\mathbb{E}}\left[Y_t\right]}
    \end{IEEEeqnarray}
  \end{proposition}
  \begin{proof}
    The parity multiplied derivatives of the exponential of the integral of the hazard rate
    satisfies the recursion-convolution lemma, generating the moments of $Y_t$, and hence is
    the characteristic function of $Y_t$.
  \end{proof}

  \section{Discussion}
  In most circumstances $\lambda$ is the new born infant mortality due to ageing alone, and
  is less than $1$ in $32000$ person-years. As such the hazard rate is very close to the
  original hazard rate observed by Gompertz. Intuitively, when $\mu \ggg \sigma^2 / 2$ the
  process becomes approximately deterministic due to the large impact of the drift.

  Conversely, the central conjecture in the preliminary material is that ageing is driven by
  stochastic accelerations, requiring that the drift vanish $\mu = 0$. In this case the
  anomalous Gompertz hazard rate simplifies to:
  \begin{IEEEeqnarray}{rCl}
    h
    & = &
    \lambda e^{t\sigma^2 / 2}
    e^{\lambda \frac{e^{t\sigma^2 / 2} - 1}{\sigma^2 / 2}}
  \end{IEEEeqnarray}
  We have observed the anomalous Gompertz hazard rate in mortality rates in Alberta while
  conducting the Hazard Rate Zoo experiment. Specifically when we remove the dominate
  exponential process of a doubling of mortality every $7$ years from an infant mortality of
  $1$ in $32000$ person-years:
  \begin{IEEEeqnarray}{rCl}
    h
    & = &
    \displaystyle\frac{2^{t/7}}{32000}
    e^\frac{7 \left(2^{t/7} - 1\right)}{32000 \ln 2}
  \end{IEEEeqnarray}
  there remains a residual anomalous growth in mortality with ageing, reflecting the higher
  order affect of the stochastic accelerations.
\end{document}